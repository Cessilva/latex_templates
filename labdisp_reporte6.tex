\documentclass{mylib/reporteConCalif}
\title{Reporte}
\author{rodrigofranciscopablo }

\subject{Laboratorio de Dispositivos y circuitos electrónicos}
\mytitle{Reporte de práctica 6}
\mysubTitle{Circuitos con diodo Zener}
\students{Francisco Pablo \textsc{Rodrigo}}
\teacher{M.I. Guevara Rodríguez \textsc{Ma. del Socorro}}
\group{8}
\deliverDate{3 de abril de 2019}
\usepackage{mathtools}
\usepackage{amsmath}
\usepackage{float}
\usepackage{tabu}
\newcommand{\insertImage}[3]{
	\begin{figure}[H]
		%\centering
		\includegraphics[width=#3cm]{#1}
		\caption{#2}
	\end{figure}
}

\begin{document}

\coverPage

%\tableofcontents
%\newpage

\section{Objetivos}

\subsection{General}

Analizar y diseñar circuitos electrónicos que contienen diodos semiconductores.

\subsection{Particular}

Analizar, diseñar, simular e implementar circuitos reguladores con diodo Zener.

\section{Introducción}

El diodo zener se puede utilizar para regular una fuente de voltaje. Este semiconductor se fabrica en una amplia variedad de voltajes y potencias.
Estos van desde menos de 2 voltios hasta varios cientos de voltios, y la potencia que pueden disipar va desde 0.25 watts hasta 50 watts o más.
La potencia que disipa un diodo zener es simplemente la multiplicación del voltaje para el que fue fabricado por la corriente que circula por él. Esto es

$$ P_z = V_z \cdot I_z$$

El cálculo del resistor $R_s$ está determinado por la corriente que pedirá la carga (lo que vamos a conectar a esta fuente de voltaje). 

Este resistor se puede calcular con la siguiente fórmula

$$R_s = \frac{V_{in min}- V_z}{1.1 \cdot I_{L max}}$$

En donde:
\begin{itemize}
	\item $V_{in min}$ es el valor mínimo del voltaje de entrada.
	\item $I_{L max}$ es el valor de la máxima corriente que pedirá la carga.
\end{itemize}	

Una vez conocido Rs, se obtiene la potencia máxima del diodo zener, con ayuda de la siguiente fórmula.

$$P_D = \frac{V_{in min}- V_z}{R_S - I_{L min}} \cdot V_z$$

\newpage
\section{Previo}

\begin{center}
    \begin{tabular}{ | p{4cm} | p{4.5cm} |p{4.5cm} |}
    \hline
    \multicolumn{3}{|c|}{Voltajes y corrientes del diodo Zener} \\
  	\hline

	 & $V_{in}$ mínimo (13 V) & $V_{in}$ máximo (20 V) \\ \hline
	
	$I_L$ mínima (1mA) & \begin{flalign*} V_Z =  \\ I_Z = \end{flalign*} & \begin{flalign*} V_Z =  \\ I_Z = \end{flalign*} \\ \hline
	
	$I_L$ máxima (10mA) & \begin{flalign*} V_Z =  \\ I_Z = \end{flalign*} & \begin{flalign*} V_Z =  \\ I_Z = \end{flalign*}\\ \hline
	
    \end{tabular}
\end{center}

\insertImage{img/labdisp_pract6/one}{Circuito con un $V_{in} = 13 V$ y $I_L = 1 mA$}{8}

\insertImage{img/labdisp_pract6/two}{Circuito con un $V_{in} = 20 V$ y $I_L = 1 mA$}{7}

\insertImage{img/labdisp_pract6/three}{Circuito con un $V_{in} = 13 V$ y $I_L = 10 mA$}{7}

\insertImage{img/labdisp_pract6/four}{Circuito con un $V_{in} = 20 V$ y $I_L = 10 mA$}{7}

\newpage
\section{Desarrollo}


\section{Conclusiones}


\end{document}