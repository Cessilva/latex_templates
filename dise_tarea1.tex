\documentclass{mylib/reporte}
\usepackage{float}
\graphicspath{ {img/dise/} }
\title{Reporte}
\author{rodrigofranciscopablo }

\subject{Diseño Digital Moderno}
\mytitle{Tarea 1}
\mysubTitle{Conceptos}
\students{
	Francisco Pablo \textsc{Rodrigo}
}
\teacher{Ing. Mandujano Wild \textsc{Roberto F.}}
\group{6}
\deliverDate{febrero 2019}


\begin{document}

\coverPage

\tableofcontents

\section{Explicar la computadora EAI 180.}
EAI 180 de Electronic Associates Incorporated, Nueva Jersey, es una llamada computadora híbrida (hybris [griego]: de doble filiación), año de fabricación 1970. Contiene las partes de una computadora análoga y éstas de una computadora digital. El dispositivo está equipado con circuitos integrados de la primera generación (tecnología DTL). El circuito de cálculo está enchufado con cables en el panel frontal. El tiempo de ciclo de la parte analógica se puede establecer en menos de 10 ms. Con ese parámetro, una ecuación se resolverá al menos 100 veces por segundo. Así que puedes ver la salida con un simple osciloscopio.

\section{¿Qué es una computadora cuántica?}

En la computación cuántica se abandonan los sistemas lógicos empleados por los sistemas computacionales que se utilizan hoy en día y se usa el modelo de los estados del átomo para realizar sus procesos.
La física cuántica tiene un comportamiento diferente a la física clásica (mecánica, termodinámica, electromagnetismo, óptica, etc.), y es que las partículas pueden adoptar estados de 1 o 0, o de 1 y 0 a la vez.
Para medir la capacidad de una computadora cuántica, se utilizan unidades de procesamiento, que son átomos individuales. En el caso de los ordenadores cuánticos, se mide en bits cuánticos o qubits. A mayor cantidad de qubits, más rápido funcionan.

Los átomos, además de tener la capacidad de adoptar diferentes estados simultáneamente, cuentan con otra cualidad que se le nombra entrelazamiento atómico.

Gracias a esto, un átomo puede transmitir propiedades a otro sin que haya algún medio de transmisión. De esta manera, cualquier cambio en el estado de uno de los átomos entrelazados, provocará en el otro un cambio automático.

Una primera versión funcionaba con 512 qubits y tenía un costo aproximado de 11 millones de euros. Una versión más reciente que se hizo pública en 2013, tiene 108 millones de veces más velocidad de procesamiento que una computadora común.

\section{¿Dónde queda Hebroville y dónde queda Zacualpan?}

Hebbronville es un lugar designado por el censo ubicado en el condado de Jim Hogg en el estado estadounidense de Texas.


El Municipio de Zacualpan es uno de los 125 municipios en que se encuentra dividido el Estado de México, ubicado en el sur de la entidad en los límites con Guerrero, su cabecera es el Real de Minas Zacualpan. Su cabecera es el primer centro minero más antiguo del continente, motivo por el cual es conocido por su actividad minera aún muy presente en la vida cotidiana.

\section{Describir la missión Apollo XI}

Las naves Apolo basaban su funcionamiento en tres elementos:
1.- El módulo de mando: de forma cónica, contiene el asiento de los astronautas y los paneles de control.
2.- El módulo de servicio: se encuentran los equipos eléctricos, depósitos de oxígeno, hidrógeno, helio y motores de maniobras.
3.- El módulo lunar: con capacidad para dos personas.

Se evaluaron muchos sitios para el lugar del alunizaje hasta que se llegó a recomendar el “Mar de la Tranquilidad”. Ese fue el lugar elegido.
Los astronautas seleccionados para realizar este viaje fueron Neil Armstrong, Edwin E. Aldrin y Michael Collins.
El día para el despegue fue el 16 de julio de 1969, la llegada a la Luna el 21 y el retorno a la Tierra el 24. En total serían 8 días.

Los tres astronautas se encontraban en una cuarentena impuesta por los médicos del proyecto. El objetivo era evitar que se contaminaran con bacterias.

Cuando realizaron una conferencia de prensa se presentaron frente a los periodistas en una jaula plástica, rodeados por fuerte corrientes de aire que formaban una especie de “muro de viento”, medidas que evitaban la transmisión de cualquier germen.

El domingo 20 de julio, ya en la órbita lunar, Aldrin y Armstrong se trasladaron al módulo “Águila”. Michael Collins cerró la compuerta y permaneció pilotando el módulo de control “Columbia”, esperando la separación de la cápsula y apoyando las maniobras del módulo lunar.

\section{¿Qué es el chilam balam}

Chilam Balam es el nombre de varios libros que relatan hechos y circunstancias históricas de la civilización maya. Escritos en lengua maya, por personajes anónimos, durante los siglos XVI y XVII, en la península de Yucatán. A ese nombre se le agrega el nombre de la población en donde fueron escritos, por ejemplo, el Chilam Balam de Chumayel, etc.

Son fuente importante para el conocimiento de la religión, historia, folclor, medicina y astronomía maya precolombina.

\section{Describir las características del Kan Balam}
"KanBalam" es una de las supercomputadoras paralelas más poderosas de América Latina con una capacidad de procesamiento de 7.113 Teraflops (7.113 billones de operaciones aritméticas por segundo). Cuenta con 1,368 procesadores (cores AMD Opteron de 2.6 GHz), una memoria RAM total de 3,000 Gbytes y un sistema de almacenamiento masivo de 160 Terabytes.

Ofrece a la comunidad académica y de investigación nacional una capacidad de cálculo sin precedente en el país.

\section{Arquitectura de la Apple II}
Las primeras unidades salieron a la venta el 5 de junio del 77 e incluían un procesador 6502 de 1 MHz, 4 KB de memoria RAM, 12 KB de ROM que incluían Integer BASIC y una interfaz para cintas de casete. La capacidad gráfica se fijó en una resolución de pantalla de 24 líneas por 40 columnas de texto en mayúsculas y una salida de vídeo compuesto NTSC para poder conectar a un TV o a un monitor.

El Apple II no solo usaba las cintas para almacenar los datos puesto que, poco después, se lanzó al mercado una unidad de disco externa (de 5,25") que se conectaba a uno de los puertos de expansión del computador y que, hoy en día, su controladora se sigue considerando todo un referente en el mundo del diseño electrónico.

Pero lo realmente significativo del Apple II es el diseño abierto por el que optó Wozniak y la inclusión de ranuras de expansión que facilitaron el desarrollo de ampliaciones y periféricos.

\section{CP/M}

CP/M (Control Program for Microcomputers) es un sistema operativo de un solo usuario/Una sola tarea desarrollado por Gary Kildall para el microprocesador Intel 8080
Si está familiarizado con MS-DOS, CP / M se siente como una versión ligeramente primitiva de DOS.

\section{¿Cómo se saca el logaritmo en cualquier base}

\insertImage{log}{Logaritmo en cualquier base}{5}

\section{¿De dónde viene la palabra dígito?}

Viene del latín dedo. Aplicado a los números que pueden contarse con los dedos.
\section{¿De dónde viene la palabra decimal?}

La palabra "decimal" se aplica al sistema de medidas que usa unidades de diez partes iguales y viene del ordinal latino decimus (décimo).

\section{Escribir 200 palabras con etimología greco-latina}

\begin{enumerate}
	\item Educar: Del latín educare, proviene de educere, que se divide en: ex: (fuera  de) y ducere (guiar, conducir).
	\item Alumno: Del latín alumnus, que viene de alére (alimentar). Persona que es alimentada.
	\item Profesor: Del latín professor. Procede del verbo profiteor, el que declara o habla delante de la gente.
	\item Escuela: Del griego scholé: ocio, tiempo libre.  El latín transformó esta palabra a schola, que usa también como ludus (juego), también entrenamiento y diversión.
	\item Universidad: Del latín universitas, que surge del adjetivo universus, todo, universal, y este viene de unus (uno).
	\item Alopatía :sujeción a influencias ajenas
	\item Ácido.(L). Acidus= agrio.
	\item Actina.(G). Aktis= rayo.
	\item Adipocito.(L). Adeps= grasa, (G) kytos= hoyo.
	\item Adrenal.(L). a= a, hacia, renal=riñones.
	\item Adsorción.(L). Ad= hacia. sorbere= chupar.
	\item Alelo.(G). Allelon= uno de otro.
	\item Alcaloide.(arabe y griego). al= el, gali= fresno, eidos= forma.
	\item Alantoína.(G). Allos= salchcha.
	\item Alergeno.(G). Allos= otro, ergon= actividad. genes= producción, descendencia.
	\item Alergia.(G). Allos= otro, ergon= actividad.
	\item Alostérico. (G). Allos= otro, stereos= sólido.
	\item Alveolo. (L). Alveolus= cavidad pequeña.
	\item Ameboide. (G). ameboide= Cambio.
	\item Amigdala. (G). Amygdale= almendra (igual que núcleo).
	\item Antígeno. (G). Anti= contra, genos= nacimiento, descendencia.
	\item Auricular. (L). Auricular= oreja pequeña.
	\item Piloro.(G). Pyloros= portero.
	\item Bacilo.(L). Bacillum= palo pequeño.
	\item Catálisis.(G). Katalepsis= disolución (en el sentido de rotura).
	\item Clavícula.(L). Clavicula= llave pequeña.
	\item Enzima.(G). Enzyme= levadura.
	\item Especie.(L). Tipo particular.
	\item Fémur.(L).  Femur= muslo.
	\item Gameto.(G). Gameto= cónyuge.
	\item Glicano.(G). Dulce.
	\item Hapteno.(G). Haptos= coger, enlazar.
	\item Hormona.(G). Hormaein= estimular.
	\item Linfa.(G). Lympha= agua.
	\item Lisis.(G). Lysis= disolución (en el sentido de rotura, destrucción)..
	\item Málico.(G). Malum= manzana. // (Ácido málico).
	\item Mitocondria. (G).Mitos= hilo, chondros= enlazar.
	\item Morfina. Proviniente del dios griego del sueño, Morfeo.
	\item Minuto.(L). Minutus= pequeño.
	\item Núcleo.(L). Nucleus= almendra.
	\item Parásito.(G). Para= al lado, sitos= alimento.
	\item Partenogénesis.(G). Parthenos= virgen, genesis=  que produce, descendencia.
	\item Patógeno.(G). Pathos= que sufre, genesis=  que produce, descendencia.
	\item Película.(G). Pellicula= piel pequeña.
	\item Péptido.(G). Peptein= digerir.
	\item Perenne.(L). Per= a través, annus= año.
	\item Piloro.(G). Pyloros= portero.
	\item Plancton.(G). Plangktos= errante.
	\item Plasma.(G). Plasma= forma.
	\item Polímero.(G). Polys= mucho, meros= parte.
	\item Polisacárido.(G). Polys= mucho, saccharum= azucar.
	\item Procariota.(G). Pro= antes, karyon= nucleo.
	\item Proteína.(G). Proteion= primero.
	\item Ptialina.(G). Ptyalon= saliva.
	\item Tropismo.(G). Trope= luz.
	\item Vacuola.(L). Vacuus= vacío.
	\item Vitamina. Termino derivado de Vita ammoniacum (goma resinosa).
	\item Zygoto.(G). Zygoto= unido.
	\item Adiáfora- Según los estoicos, acciones que son moralmente indiferentes.
	\item Adikia- Injusticia.
	\item Aphrosyne (Afrosine)- Lo que es tonto o ridículo.
	\item Ágape- Sentimiento de estima o afecto general. Amor del alma. Amor fraternal o universal y desinteresado (en el Cristianismo).
	\item Ágathos- Lo bueno.
	\item Agatón- El Bien.
	\item Agnosia- Actitud de quien afirma no saber nada.
	\item Ánemos- Alma como principio vital. Aire, viento.
	\item Átomo- No divisible (deriva de “a” (no) y “tomos” (cortar)). Es el fundamento del concepto materialista de Leucipo y Demócrito que surgió para contrarrestar a la imposibilidad del movimiento propuesta por la Escuela Eleática.
	\item Dialéctica- En general el término se refería a un “discurso” o “debate”, un intercambio de ideas en el que los interlocutores -que tratan de defender sus propias convicciones- también comparten algunos principios básicos.
	\item Tetralema- Significa “cuatro proposiciones” (“tetra” es cuatro y “lema” es proposición en griego).
	\item Citología: el estudio de las células. Encuentra su origen en la palabras griegas “kytos" la cual significa célula y "lógos" cuya traducción puede ser: pensamiento, saber o estudio.
	\item Filosofía: el amor por el saber o el amor por la sabiduría. Deriva etimológicamente de dos palabras griegas: la primera es “filos” que se traduce en amante y “sofía" que significa sabiduría.
	\item Biblia: del griego “biblion” que significa libro o en plural libros.
	\item Caligrafía: bella escritura, escritura hermosa o escribir con letra entendible y legible. Deriva de las palabras griegas “kallos” que significa belleza y “grafein” cuya traducción es escribir.
	\item Carisma: de tener encanto o gracia para agradar a la mayoría. Deriva de la palabra griega "kharis" que significa gracia o encanto.
	\item Casa: proviene del griego “kasas”, cuya traducción puede ser: cabaña, choza o barraca.
	\item Cinética: referente a movimiento. Proviene de la palabra griega "kinesis" que se traduce en movimiento.
	\item Hemorragia: flujo excesivo de sangre. También, deriva de la combinación de dos palabras griegas: "hémos" que significa sangre y "rrhagia" que se traduce en flujo excesivo.
	\item Hemorragia: flujo excesivo de sangre. También, deriva de la combinación de dos palabras griegas: "hémos" que significa sangre y "rrhagia" que se traduce en flujo excesivo.
	\item Homólogo: de igual o semejante, que piensan u opinan de manera igual o semejante. Proviene de las palabras “homo” que significa mismo o igual y de "lógos" cuya traducción puede ser: pensamiento, saber o estudio.
	\item Mitología: el estudio de las historia, leyendas, fábulas y misterios. Del griego "mythos" que significa historia, fábula o leyenda, relacionado con lo sagrado de las culturas.
	\item Nefrología: ciencia avocada al estudio de los riñones. Proviene de dos palabras griegas: "nephros" cuyo significado es riñón y "lógos" que por lo general se traduce en: pensamiento, saber o estudio.
	\item Odontología: ciencia avocada al estudio de los dientes. Encuentra su origen griego en las plabras: "odónto" que significa diente y "lógos" cuya traducción puede ser: pensamiento, saber o estudio.
	\item Pediatría: rama o especialidad de la medicina que se encarga del estudio y cuidado de la salud de los niños/as. Deriva de dos palabras griegas: "pasis o paidós" que significa niño y de "iatrós" cuya traducción es médico.
	\item Psicología: la ciencia que se dedica al estudio de las facultades del alma humana, en la actualidad está muy vinculada al estudio del cerebro humano. El origen de esta palabra de dos palabras o voces griegas "psique" que viene a ser el significado de alma y de "logos", cuya traducción puede ser: pensamiento, saber o estudio.
	\item Técnica: término que se emplea para designar o dar un nombre a lo que el ser humano es capaz de producir o desarrollar a partir de sus capacidades y conocimiento. Relacionada con las palabras griegas griego "téchne o tekné", cuya traducción es técnica.
	\item Terapia: es todo aquel tratamiento aplicado para rehabilitación y curación. Deriva del griego "therapia" que significa tratamiento o curación.
	\item Tóxico: referente a veneno. deriva del griego "toxik" que se traduce en veneno de flechas.
	\item Xenofobia: de manifestar miedo, rechazo o temor a lo extranjero o foráneo. Deriva de la combinación de dos palabras griegas: "xénos" que significa extranjero y "phóbos" que traduce como se mencionó anteriormente en: miedo, temor o rechazo.
	\item Aeropuerto (aero =aire + puerto  Lugar resguardado del viento a la orilla del mar o de un río)  Es  un  terminal en tierra donde se inician y concluyen
	\item los viajes de transporte aéreo en aeronaves.
	\item Cardiólogo  (cardia) corazón y del sufijo “logía” del griego “logia” que indica estudio cardiología = Rama de la medicina que se especializa en enfermedades del corazón.
	\item Biólogo (palabra  que combina las palabras griegas (bios=vida) y logos (tratado, ciencia, discurso). Persona que estudia la vida.
	\item Radiología procede del latín y del griego, del  sustantivo latino “radius”, que puede traducirse como “rayo”.La palabra griega “logia”, que es sinónimo de “estudio”.
	\item Hidráulico (hidráulica» proviene del griego  (hydraulikós) que, a su vez, viene de «tubo de agua», palabra compuesta  («agua») («tubo»).
	\item Ecología proviene del griego, oikos = casa y logos = ciencia. Es por lo tanto, la ciencia que estudia las relaciones de los seres vivos entre sí y con su hábitat y la naturaleza.
	\item Laicismo (laiko=pueblo) voz que se oponía al clérigo.
	\item Aguas. Etimología: del latín aqua, "agua".
	\item Aurora. Etimología: del latín Aurora, con el mismo significado.En la mitología romana, diosa del amanecer.
	\item Dios. Etimología: del latín deus, con el mismo significado
	\item Escritura. Etimología: del latín scriptura, con el mismo significado.
	\item Leguminosas. Etimología: del latín legumen, "legumbre".
	\item Marte. Etimología: del latín Mars, dios romano.
	\item Tierra. Etimología: del latín terra, "tierra"
	\item Navidad. Etimología: del latín nativitas, "nacimiento"
	\item Abastar. Etimología: del bajo latín bastare, de bastus, "suficiente.
	\item Abdominal. Etimología: del latín abdominalis, con el mismo significado.
	\item Álgebra: Procede del árabe "al-jabr", que significa recomponer o reconstruir. Hacia el siglo IX de nuestra era, el matemático árabe Al-Khowarizmi escribe una de las obras más importantes de la época, "Kitab al-jabr wa al-muqabalah" que dio lugar al nombre de esta disciplina. Se trataba en el libro...
	\item Aritmética: Del griego ariqmoV (arithmós) que significa número.
	\item Cálculo: Del latín "calculus", que significa guijarro o piedra pequeña. Antiguamente se utilizaban para contar o realizar operaciones, para "calculare". Todavía hoy se usa la palabra cálculo para llamar a las piedras que se forman en algunos órganos del cuerpo como el riñón.
	\item Cateto: Palabra de origen griego que significa lo que cae perpendicularmente
	\item Cero; cifra: La palabra cifra proviene del árabe sifr, que significa vacío o cero. Más tarde su significado se generalizó usándose para designar a todos los numerales. Para designar al cero se tomó del italiano la palabra zero, derivada a su vez del latín zephyrum.
	\item Exponente: De la preposición latina "ex" que significa fuera, y del verbo "ponere" que significa poner o colocar. Por lo tanto, poner fuera. Véanse otras construcciones semejantes en palabras como exculpar, excéntrico, expectorar, exhumar...
	\item Fracción: Del latín fractio, nis, derivada de frango, quebrar, hacer pedazos; es decir, cosa rota, quebrada.
	\item Geometría: Del griego "geo" que significa tierra, y "metria" que significa medir. Esto tiene mucho sentido porque el origen de la geometría puede situarse en Egipto, cuando los empleados del faraón tuvieron que utilizar determinadas técnicas de medición para calcular la extensión de terreno que poseía cada agricultor.
	\item Hexágono (con hache): Procede del griego "exa" y "gonos" que significa seis ángulos. La Real Academia Española admite esta palabra con hache y sin ella. Desde mi punto de vista debería fomentarse el uso de la forma con hache. En griego la palabra exa lleva un espíritu áspero en la e que ha dado lugar a la hache del castellano.
	\item Integral: El término integral se debe a Jacques Bernoulli. En un principio Leibniz llamó a las dos ramas del cálculo que había inventado calculus differentialis (las tangentes las obtenía estudiando el comportamiento de pequeñas diferencias de las variables) y calculus summatorius (las áreas de superficies las obtenía mediante sumas de pequeñas áreas). Después Johann Bernoulli le sugirió que sería mejor llamar a este último calculus integralis, cosa con la Leibniz se mostró de acuerdo. Sí, he dicho Johann, aunque el primero en usarlo fue su hermano Jacques, aunque Johann decía que el término se debía a él mismo... Estos Bernoulli siempre a la gresca.
	\item Línea: Proviene del latín linea, que quería decir "hilo de lino", derivado de linum, "lino". Los griegos decían linon para indicar "lino" o "cosa hecha de lino" o directamente "hilo".
	\item Logaritmo: Del griego "logos" que significa razón, proporción, manera o relación y de "arithmos" que significa número. Indica por tanto la relación que hay los términos (números) de una progresión aritmética y los de una geométrica. Al principio, Neper llamo a los logaritmos números artificiales.
	\item Matemáticas: Procede del verbo griego "mánthano", que significa aprender, pensar, aplicar el espíritu. A partir de ahí se forma el sustantivo "máthema", que significa conocimiento, y de éste el adjetivo "mathematikós". En el latín se adopto la forma "mathematicus". El significado de la palabra matemáticas sería entonces aquello que se piensa y se aprende, y el matemático es aquel que piensa, que aprende, que aplica el espíritu. El hecho de que sea frecuente utilizar este término en plural es porque en latín "mathematica" es un sustantivo plural. También se ha dicho que se prefiere el término en plural porque abarca a una serie de disciplinas, como son la geometría, el álgebra, el análisis, la topología, la estadística, etc. Platón opinaba que nadie podía considerarse educado si no tenía conocimientos de matemáticas.
	\item Paradoja: Del griego paradoxoV (para y doxos) y significa más allá de lo verdadero o creíble.
	\item Polígono: Del griego poligwnoV (poli y gonos) que significa muchos o varios ángulos.
	\item Primo: Referido a los números que sólo tienen dos divisores. Viene del griego prwtoV que significa primero, y alude a la propiedad, conocida como teorema fundamental de la aritmética, que tiene todo número de obtenerse como producto de números "primeros"; es decir, los "primeros" son los que "producen" todos los otros.
	\item Punto: Del latín pŭnctum, punzada, picadura.
	\item Seno: Esta palabra surgió de una traducción equivocada. Los hindúes utilizaron la palabra "jiva" para designar a la semicuerda que hoy conocemos como seno. Los árabes adoptaron para este concepto la palabra "jiba". Cuando Roberto de Chester traducía una obra del árabe se encontró con la palabra técnica "jiba", desconocida para él, y la confundió con la palabra "jaib" que significa bahía o ensenada (recordemos además que los árabes omiten las vocales al escribir). Así que "jiba" fue traducida por la palabra latina "sinus" que significa curva hueca, bahía o ensenada.
	\item Trivial: Me encanta esta palabra: cuando un matemático no tiene ganas de explicar algo, bien porque le resulta insoportablemente aburrido, bien porque ha quedado para cenar, o bien porque no sabe muy bien cómo explicarlo, va y dice algo así como "la demostración de este hecho es trivial" o "es un ejercicio trivial demostrar que...", expresión con la que deja a todo el mundo con un palmo de narices, porque nadie se atreve a preguntar algo que resulta ser trivial. Su etimología es prácticamente... trivial: viene del latín trivium, que es la palabra que utilizaban los romanos para nombrar el lugar donde se encontraban tres caminos o vías. ¿Y qué es lo que pasaba en esos lugares? Pues que la gente se encontraba. ¿Y de qué habla la gente cuando se encuentra? Pues de cualquier cosa sin importancia, del tiempo, de la comida, de lo que es sabido por todos, de trivialidades.
	\item Educar: Del latín educare, proviene de educere, que se divide en: ex: (fuera  de) y ducere (guiar, conducir). Educar vendría siendo guiar a la persona para que saque lo mejor de sí, para que desarrolle todo su potencial.
	\item Alumno: Del latín alumnus, que viene de alére (alimentar). Persona que es alimentada. En un principio, alumnus se refería básicamente al niño que por instinto biológico se alimenta del pecho de su madre. Con el tiempo pasó a entenderse como la persona que se alimenta de conocimiento.
	\item Estudiante: Esta palabra se forma del sustantivo latino studium: estudio, que proviene del verbo studeo, estudiar. En un principio, en latín, estudiar se refería a algo que la persona dedicaba especial atención, y en griego significaba esforzarse mucho en alguna actividad. Más adelante, estudiar pasó a ser lo que hoy conocemos: “Ejercitar el entendimiento para alcanzar o comprender algo” (RAE).
	\item Maestro: Proviene del latín magister (en acusativo magistrum), el que más sabe y por lo tanto dirige a los otros. Magis es ‘más’, y ter es un sufijo indoeuropeo que que contrasta dos opuestos, por ejemplo: magister-minister-maestro-ministro.
	\item Minis es ‘menos’ y magis es ‘más’. Por tanto el maestro es el que tiene mayor conocimiento y el ministro el que menos, por eso el maestro podía ser jefe del ministro. Si bien antes, al que era maestro de escuela, se lo llamaba ‘litterator’. Fue con los años que se adoptó el término a la pedagogía.
	\item Profesor: Del latín professor. Procede del verbo profiteor, el que declara o habla delante de la gente. Terminaría por significar el que habla delante de los alumnos.
	\item Docente: Del latín docens, el que enseña, que viene del verbo docere, enseñar. Docere tiene la raíz indoeuropea dek, que signfica pensamiento o aceptación.
	\item Gimnasio: del latín gymnasium, del griego gymnásion. Lugar para hacer ejercicio. La palabra ha ido perdiendo su otro significado: lugar de enseñanza, que ya no se entiende así si no se relaciona con el nombre de un colegio.
	\item Escuela: Del griego scholé: ocio, tiempo libre.  El latín transformó esta palabra a schola, que usa también como ludus (juego), también entrenamiento y diversión. El tiempo libre y el ocio era entendido por los griegos como el espacio para cultivarse, para aprender, en vez de ocuparse de otras labores que no lo alimentaban.
	\item Universidad: Del latín universitas, que surge del adjetivo universus, todo, universal, y este viene de unus (uno). En sus inicios las universidades se referían a instituciones, agremiaciones, corporaciones, pero no a un lugar que necesariamente concentrará personas a estudiar. Desde el Renacimiento comenzó a tener el significado de institución de educación superior.
	\item Siesta. La palabra siesta tiene una curiosa procedencia. Una de las Reglas de San Benito consistía en guardar reposo después de la ‘sexta hora’ latina, para nosotros el mediodía, que es la hora de más calor. Aquí tiene su origen la palabra ‘sextear’ o ‘guardar la sexta’, que después se deformó popularmente en ‘sestear’ o ‘guardar la siesta’.
	\item Restaurante. La palabra restaurante viene de restaurant, que significa ‘restaurativo’. En 1765, un mesonero llamado Boulanger abrió una casa de comidas en cuya fachada colgó un eslogan en francés que en español venía a decir: ‘Venir a mí todos los de estómago cansado y yo los restauraré”. Tal fue su éxito que todas las cosas de comidas pasaron a llamarse restaurants y los cocineros restauradores.
	\item Santiamén. Esta palabra tiene su origen en el aburrimiento, pues cuando las misas se oficiaban en latín la gente esperaba ansiosamente la bendición del padre para irse lo más rápido posible de la iglesia. La expresión de ‘en un santiamén’ se refiere a la última parte de la oración In nomine Patris, et Filli, et Spiritus Sancti. Amén.
	\item Té. Esta palabra tiene su origen en China, ya que el símbolo de esta palabra proviene de la unión de la palabra hierba, chá, y del árbol, tú, y en la provincia china de Fujian se pronuncia ‘Tay’, y fue allí donde los importadores de té en Europa, los holandeses, aprendieron a pronunciarla. Con el tiempo, la palabra ‘tay’ se convirtió en té para España, tee para los alemanes, thé para los franceses y tea para los ingleses.
	\item Jamón. Esta palabra es una evolución de la palabra ‘pierna’ en la ramificación francesa del latín, es decir, de jambe o jambon, pero fue mucho después cuando se popularizó en España como jamón, pues el primer escrito data de 1335. Antes de esto se referían a este alimento como pernil, algo totalmente en desuso.
	\item Esnob/ Snob. A pesar de que hay varias versiones sobre su origen, se dice que tiene lugar en la elitista escuela Eton College, en Inglaterra. Según cuenta la historia, para diferenciar a los pocos alumnos que no eran nobles, se les calificaba con la abreviación latina ‘S.Nob’ (sine nobilitate, sin nobleza). Ahora, la palabra se utiliza para nombrar a aquellas personas que desean aparentar la pertenencia a una clase social superior.
	\item Usted. El origen de esta palabra tiene que ver con el término ‘vos’, como solía tratarse al Rey y otras autoridades en los siglos XVI y XVII. Cuando los españoles viajaron a América utilizaron el ‘vos’ en algunos países hispano hablantes. Al cabo del tiempo, se comenzó a usar por las clases sociales más bajas y fue necesario otros términos para resaltar el respeto. Ahí apareció en uso de ‘vuestra merced’, que derivó en ‘vueced’ y al final se convirtió en ‘Usted’. Poco a poco, ‘vos’ quedó en desuso en España.
	\item Ok. El origen de esta palabra tiene dos versiones. Por un lado, en 1839 el periódico Boston Morning Post, como broma, escribía mal las iniciales y entre paréntesis explicaba su significado. Uno de estos fue el uso del OK, y entre paréntesis decía (all correct). La otra versión hace referencia a ‘cero muertos’ (0 killed), una expresión usada por los soldados norteamericanos cuando no había bajas.
	\item Ojalá. Proviene del árabe, como muchas palabras del español, debido a la influencia de la invasión musulmana en el siglo VIII. Se trata de la evolución de ‘aw šá lláh, que significa en árabe ‘si dios quiere’.
	\item Borracho. Esta palabra proviene de burra, que en latín significa hez o sedimento, y hace referencia a la devaluación de la condición que tiene una persona bajo los efectos del alcohol.
	\item Chicle. Curiosamente, la palabra chicle hace referencia a la ciudad donde se creó la primera fábrica de goma de mascar en suelo norteamericano, la pequeña ciudad de Chicle, en Colorado.
	\item Láser. Aunque ya está totalmente aceptada como palabra, lo cierto es que está formada por las siglas Light Amplification by Stimulated Emission of Radiation, que significa amplificación de luz estimulada por emisión de radiación.
	\item Mambrú. ¿Mambrú se fue a la guerra? Parece ser que esta palabra es una deformación a la hora de pronunciar el nombre del General Marlborough, del siglo XVIII. La canción que conocemos se hizo popular debido a que una nodriza se la cantaba a uno de los hijos de María Antonieta.
	\item Dabuten. A pesar de que éste término no está recogido en los diccionarios oficiales sí que es utilizado, sobre todo, por jóvenes. Sí que se reconoce la palabra bute, que significa ‘mucho’ y junto a la preposición ‘de’ se utiliza para referir algo que vale mucho, de calidad. Como muchas otras palabras, su uso cotidiano hizo que se popularizara y comenzara a evolucionar en palabras similares, como ‘debute’, ‘debuti’, ‘debuten’. Hoy en día se utiliza para decir algo que es genial.
	\item Caca. Proviene del griego, kakos, que significa ‘cosa mala’.
	\item Olé. Hay muchas hipótesis sobre el origen de esta palabra. Hay quien afirma que viene del verbo griego ololizin, utilizado como grito de júbilo. Otros dicen que viene de la Biblia, cuando Jacob es engañado en su boda con Raquel, pues la gente intentaba avisarle de que se trataba de Lea y no de su amada, diciendo ¡Oh, Lea! Pero la hipótesis más extendida tiene que ver con el árabe con la expresión Allah (Oh, Dios). La RAE, por otra parte, recoge que ¡olé! proviene de la exclamación árabe Wa-(a)llah (¡Por Dios!), una exclamación de entusiasmo ante una belleza o alegría sorprendente o excesiva. En el idioma árabe, no existe la vocal “e” y, en ocasiones, la vocal “a” suena parecido a la “e”.
	\item Bidé. ¿Dudas de cómo se utiliza éste accesorio del baño? Pues quizás su origen te dé alguna pista. Esta palabra viene del francés bidet, que significa caballito.
	\item Chándal. Esta palabra común también viene del francés, chandail, que hace referencia al jersey utilizado por los los vendedores de verdura.
	\item Macarrón. Curiosamente, esta palabra proviene del italiano maccarone y éste a su vez del griego, es decir, ‘comida funeraria’.
	\item Trabajar. ¿Es para ti trabajar un suplicio? Lo curioso es que así lo veían antiguamente, pues la palabra viene del latín vulgar tripaliare, un instrumento de tortura compuesto por tres palos de madera.
	\item Guiri. Éste término es utilizado para designar a los turistas podría venir de la palabra guiri-gay, un vocablo que alude a un lenguaje difícil de entender. Sin embargo, hay quien dice que es debido a los primeros turistas extranjeros que vinieron a España, pues éstos preguntaban ‘Where is…?’ y por ello se ganaron el término de guiris.
	\item Calendario. Del latín “kalenadarium” que significaba registro o libro de cuentas de un prestamista.
	\item Candidato. Del latín “canditatus” y de este “cadidum” (blanco) ya que cualquier persona que aspirará a un cargo político debía ser intachable
	\item Acromion: Apófisis del omóplato, con la que se articula la extremidad externa de la clavícula. 'extremo', 'punta' + 'hombro'.
	\item Adenoides: Tejido ganglionar que existe normalmente en la rinofaringe. ‘glándula’ + ‘que tiene el aspecto de’.
	\item Anatomía: Estudio de la disección. Tiene por objeto estudiar el número, estructura y situación de las diferentes partes del cuerpo de los animales o de las plantas.'por completo', 'por partes' + 'corte', 'incisión quirúrgica'.
	\item Apófisis: Parte saliente del hueso para articulación o inserción muscular. apo.'a partir de' +. 'cosa crecida'.
	\item Aponeurosis. Membrana de tejido conjuntivo que envuelve los músculos. 'a partir de' + ‘tendón’.
	\item Astrágalo: Uno de los huesos del tarso, que está articulado con la tibia y el peroné.  'huesecillo', 'vértebra'.
	\item Átlas:, 'Atlas' (dios que soporta el cielo) 'primera vértebra cervical'.
	\item Brazo:   Miembro del cuerpo que comprende desde el hombro hasta la extremidad de la mano. corto.
	\item Cadera: Huesos superiores de la pelvis. 'asiento', 'sentarse'.
	\item Carpo: Conjunto de huesos que, en número variable, forman parte del esqueleto de las extremidades anteriores de los reptiles y mamíferos, y que por un lado está articulado con el cúbito y el radio y por otro con los huesos metacarpianos. muñeca.
	\item Cuboides: Uno de los huesos del tarso, situado en el borde externo del pie. (Tiene forma de cubo). ‘cubo’ + ‘que tiene el aspecto de’.
	\item Deltoides: Músculo propio de los mamíferos, de forma triangular, que en el hombre va desde la clavícula al omóplato y cubre la articulación de este con el húmero.'letra  de forma triangular' 'que tiene el aspecto de'.
	\item Falange: Cada uno de los huesos de los dedos. Se distinguen con los adjetivos ordinales primera, segunda y tercera, comenzando a contar desde el metacarpo o el metatarso. 'rodillo', 'falange'.
	\item Glúteo: Músculo de la nalga. 'trasero'.
	\item Isquion: Hueso que en los mamíferos adultos se une al ilion y al pubis para formar el hueso de la cadera.
	\item Menisco: Cartílago de forma semilunar y de espesor menguante de la periferia al centro; forma parte de la articulación de la rodilla. 'mes', 'luna creciente'. Metatarso: Conjunto de huesos largos que forman parte de las extremidades posteriores de los batracios, reptiles y mamíferos, y que por un lado están articulados con el tarso y por el otro con las falanges de los dedos del pie.'después de', 'posterior' , 'tarso'.
	\item Olécranon: Apófisis del cúbito que forma el saliente del codo. 'codo' + 'cabeza'.
	\item Pericardio: Envoltura del corazón, que está formada por dos membranas: una externa y fibrosa, y otra interna y serosa.  'alrededor de' + 'corazón'.
	\item Peroné: Hueso largo y delgado de la pierna, detrás de la tibia, con la cual se articula.'cruzar', 'atravesar'.
	\item Torácico: Perteneciente o relativo al tórax. 'tórax'.
	\item Xifoides: Cartílago o apéndice cartilaginoso y de figura algo parecida a la punta de una espada, en que termina el esternón del hombre. 'espada' + 'que tiene el aspecto de'.
	\item ENFERMEDADES Y PATOLOGÍAS
	\item Osteocondrosis: Formación de tejido cartilaginoso. 'cartílago' + Hueso.+ Formación.
	\item Acné: Enfermedad de la piel caracterizada por una inflamación crónica de las glándulas sebáceas, especialmente en la cara y en la espalda. ‘extremo’, ‘punta’.
	\item Adenopatía: Enfermedad de las glándulas en general, y particularmente de los ganglios linfáticos padecimiento, sentimiento. Propio de la mononucleosis infecciosa.
	\item Aerofagia: Deglución espasmódica de aire seguida de eructos. 'aire' + 'comer'.
	\item Afasia: Pérdida del habla a consecuencia de desorden cerebral.  'no' 'decir, hablar'.
	\item Amigdalitis: Inflamación de las amígdalas. 'almendra', +  'inflamación'.
	\item Amnesia: Pérdida o debilidad notable de la memoria. 'no' + 'recuerdo'.
	\item Anafilaxia. Estado de hipersensibilidad o de reacción exagerada a la nueva introducción a una sustancia extraña, que al ser administrada por primera vez provocó reacción escasa o nula.'por completo', 'por partes' +  ‘guardián’.
	\item Anemia: Empobrecimiento de la sangre por disminución de su cantidad total o cantidad de hemoglobina. 'no' + 'sangre'.
	\item Anorexia: Falta anormal de ganas de comer. 'no' +  'apetecer'.
	\item Apoplejía: Suspensión súbita de la acción cerebral por derrame sanguíneo. 'sin' + 'ataque paralizante'.
	\item Arritmia: Irregularidad y desigualdad en las contracciones del corazón. 'no' + 'cadencia, ritmo'.
	\item Artritis: Inflamación de las articulaciones.  'articulación' +, 'inflamación'.
	\item Asma: Enfermedad de los bronquios, caracterizada por accesos ordinariamente nocturnos e infebriles, con respiración difícil y anhelosa, tos, expectoración escasa y espumosa, y estertores sibilantes. ‘jadeo’.
	\item Autismo: Concentración habitual de la atención de una persona en su propia intimidad, con el consiguiente desinterés respecto del mundo exterior.‘que actúa por sí mismo o sobre sí mismo + ismos. afección.
	\item Bradicardia: Ritmo cardíaco más lento que el normal. 'lento' + 'corazón'.
	\item Braquialgia: Dolor neurálgico en los brazos. 'brazo' +  'dolor'.
	\item Bronquitis: Inflamación aguda o crónica de la membrana mucosa de los bronquios.   'bronquio' +. 'enfermedad', 'inflamación'.
	\item Bulimia: Enfermedad que consiste en tener uno tanta gana de comer, que con nada se ve satisfecho.  'buey', 'vaca' + 'hambre'.
	\item Cardiopatía: Enfermedad del corazón. . 'corazón' + 'enfermedad'.

\end{enumerate}


\end{document}
