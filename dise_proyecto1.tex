\documentclass{mylib/reporte}
\usepackage{float}
\title{Reporte}
\author{rodrigofranciscopablo }

\subject{Diseño Digital Moderno}
\mytitle{Proyecto 1}
\mysubTitle{Circuito de NANDs y NORs}
\students{
	Francisco Pablo \textsc{Rodrigo}
}
\teacher{Ing. Mandujano Wild \textsc{Roberto F.}}
\group{6}
\deliverDate{23 de marzo de 2019}

\newcommand{\insertImage}[2]{
	\begin{figure}[H]
		\centering
		\includegraphics[width=#2cm]{#1}
	\end{figure}
}

\begin{document}

\coverPage

\tableofcontents
\newpage

\section{Introduction}

Las Compuertas Lógicas son circuitos electrónicos conformados internamente por transistores que se encuentran con arreglos especiales con los que otorgan señales de voltaje como resultado o una salida de forma booleana, están obtenidos por operaciones lógicas binarias (suma, multiplicación). También niegan, afirman, incluyen o excluyen según sus propiedades lógicas. Estas compuertas se pueden aplicar en otras áreas de la ciencia como mecánica, hidráulica o neumática.\\

\subsection{Compuerta NAND}
También denominada como AND negada, esta compuerta trabaja al contrario de una AND ya que al no tener entradas en 1 o solamente alguna de ellas, esta concede un 1 en su salida, pero si esta tiene todas sus entradas en 1 la salida se presenta con un 0.

\insertImage{img/dise_proy2/semisum}{6}

\subsection{Compuerta NOR}

Así como vimos anteriormente, la compuerta OR también tiene su versión inversa. Esta compuerta cuando tiene sus entradas en estado 0 su salida estará en 1, pero si alguna de sus entradas pasa a un estado 1 sin importar en qué posición, su salida será un estado 0.


\insertImage{img/dise_proy2/fullsum}{6}

\section{Materiales}

\begin{enumerate}
	\item 2 LEDs
	\item Alambre de seis colores distintos
	\item 2 compuertas NAND
	\item 1 compuerta NOR
	\item 2 dip-switch (4 posiciones)
\end{enumerate}	

\section{Planteamiento}

\subsection{Circuito de solo NANDs}
\subsection{Circuito de solo NORs}

\subsubsection{Diagrama lógico}

\insertImage{img/dise_proy2/compLog}{10}

\section{Patigrama}

\insertImage{img/dise_proy2/patigrama}{11}

\section{Implementación}

Después de algunas cuantas horas de arduo esfuerzo y de algunos cuantos errores obtuvimos el siguiente circuito.

\insertImage{img/dise_proy2/implementacion}{15}


\end{document}
