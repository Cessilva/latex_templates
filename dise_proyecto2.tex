\documentclass{mylib/reporte}
\usepackage{float}
\title{Reporte}
\author{rodrigofranciscopablo }

\subject{Diseño Digital Moderno}
\mytitle{Proyecto 2}
\mysubTitle{Sumador de 4 bits}
\students{
	Flores Martínez \textsc{Emanuel}\\
	Francisco Pablo \textsc{Rodrigo}\\
	García Ruíz \textsc{Andrea Guadalupe}
}
\teacher{Ing. Mandujano Wild \textsc{Roberto F.}}
\group{6}
\deliverDate{16 de marzo de 2019}

\newcommand{\insertImage}[2]{
	\begin{figure}[H]
		\centering
		\includegraphics[width=#2cm]{#1}
	\end{figure}
}

\begin{document}

\coverPage

\tableofcontents
\newpage

\section{Introduction}

Un sumador es un circuito que realiza la suma de dos palabras binarias.  En muchas computadoras y otros tipos de procesadores se utilizan sumadores en las unidades aritméticas lógicas. También se utilizan en otras partes del procesador, donde se utilizan para calcular direcciones, índices de tablas, operadores de incremento y decremento y operaciones similares.\\

Básicamente podemos clasificar a los sumadores en dos grandes categorías.

\subsection{Semi-sumador}
Es un dispositivo capaz de sumar dos bits y dar como resultado la suma de ambos y el acarreo. La tabla de verdad correspondiente a esta operación sería:

\insertImage{img/dise_proy2/semisum}{6}

\subsection{Sumador completo}

 Presenta tres entradas, dos correspondientes a los dos bits que se van a sumar y una tercera con el acarreo de la suma anterior. Y tiene dos salidas, el resultado de la suma y el acarreo producido. Su tabla de verdad será:

\insertImage{img/dise_proy2/fullsum}{6}

\section{Materiales}

\begin{enumerate}
	\item 5 LEDs
	\item Alambre de seis colores distintos
	\item 2 compuertas XOR
	\item 2 compuertas AND
	\item 1 compuerta OR
	\item 2 dip-switch (4 posiciones)
\end{enumerate}	

\section{Planteamiento}

Diseñar un sumador de 2 palabras y de 4 bits con CLA (Carry Look Ahead) con compuertas lógicas. No se permite el
uso de circuitos integrados sumadores ni CLA ya diseñados, el alumno debe crear su propia implementación de cada
uno de los elementos mencionados (Sumador y Carry Look Ahead).

\section{Diagrama lógico}

\insertImage{img/dise_proy2/compLog}{10}

\section{Patigrama}

\insertImage{img/dise_proy2/patigrama}{11}

\section{Implementación}

Después de algunas cuantas horas de arduo esfuerzo y de algunos cuantos errores obtuvimos el siguiente circuito.

\insertImage{img/dise_proy2/implementacion}{15}

\section{Manual de usuario}

Para el uso del sumador debemos tomar en cuenta qeu todo esta binario. EL resultado se expresa por medio de los LEDS.
Es decir, por ejemplo, podemos poner el número 1 en en el primer y segundo dipswitch y al sumarlo obtendremos dos que en binario sería 10 en este caso el número nos queda como 00010 (todos los leds apagados excepto el segundo).

\end{document}
