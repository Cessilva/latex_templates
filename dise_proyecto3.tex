\documentclass{mylib/reporte}
\graphicspath{ {img/dise_proy3/} }
\usepackage{multicol}
\usepackage{float}
\title{Reporte}

\subject{Diseño Digital Moderno}
\mytitle{Proyecto 3}
\mysubTitle{Contador BCD - GRAY}
\students{
	Flores Martínez \textsc{Emanuel}\\
	Francisco Pablo \textsc{Rodrigo}\\
	García Ruíz \textsc{Andrea}
}
\teacher{Ing. Mandujano Wild \textsc{Roberto F.}}
\group{6}
\deliverDate{4 de Junio 2019}

\begin{document}

\coverPage

\section{Introducción}


Un contador es un circuito digital capaz de
contar sucesos electrónicos, tales como
impulsos, avanzando a través de una
secuencia de estados binarios.

\section{Materiales}

\begin{enumerate}
	\item 1 74LS86
	\item 1 74LS08
	\item 2 74LS157
	\item 2 74LS73
	\item 4 LED'S
	\item C.I. 555
	\item 1 dip-switch (4 posiciones)
	\item Alambre calibre 22
\end{enumerate}

\section{Códigos}

\begin{multicols}{2}



	\subsection{Código BCD}

	\begin{tabular}{c c}
	Decimal & BCD \\
	\hline
	0 & 0000 \\
	1 & 0001 \\
	2 & 0010 \\
	3 & 0011 \\
	4 & 0100 \\
	5 & 0101 \\
	6 & 0101 \\
	6 & 0110 \\
	7 & 0111 \\
	8 & 1000 \\
	9 & 1001 \\
	\end{tabular}

	\subsection{Código Gray}

	\begin{tabular}{c c}
	Decimal & GRAY \\
	\hline
	0 & 0000 \\
	1 & 0001 \\
	3 & 0011 \\
	2 & 0010 \\
	6 & 0110 \\
	7 & 0111 \\
	5 & 0101 \\
	4 & 0100 \\
	12 & 1100 \\
	13 & 1101 \\
	\end{tabular}

\end{multicols}

\section{Descripción de la solución}

Para realizar el proyecto tuvimos que adaptar el contador BCD up/down de 3 bits
de la siguiente imagen

\insertImage{contador}{Contador up/down de 3 bits}{12}

Los contadores de n bits son fácil de implementar, pues solo tenemos que agregar
la misma estructura al final del último flip-flop.
Para hacer que cuente en GRAy, lo que hicimos fue traducir de código BCD a código GRAY
mediante compuertsa XOR. Finalmente para que el usuario pudiera decidir en entre usar
código GRAY o BCD, usamos multiplexores.


\end{document}
