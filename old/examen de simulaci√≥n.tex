\documentclass[10pt,a4paper]{article}
	\usepackage{miconfiguracion}
		\configPage
\begin{document}
	\BgThispage
	\makemytitle{Examen Simulación}{Universidad}
	\datosalu
	%\marcaagua
	\section{Matemáticas}
	%SECCION DE PREGUNTAS
	\begin{multicols*}{2}
	
	\pregunta{Simplifica la siguiente expresión
			$\dfrac{(a^2+b)^{3/2}}{a^2+b}$}
			{null}
		\opciones{$\dfrac{a^3+b^{3/2}}{a^2+b}$}
				{$(a^2+b)^{-1/2}$}
				{$\sqrt{a^2+b}$}
				{$(a^2+b)^3$}
	\pregunta{Al desarrollar el binomio 
			$(x-y)^2$}
			{null}
		\opciones{$x^2-y^2$}
				{$x^2-xy+y^2$}
				{$x^2-2xy+y^2$}
				{$x^2+2xy+y^2$}
	\pregunta{El producto 
			$(\dfrac{x^2-3x}{x-1})(\dfrac{x-1}{x-3})$}
			{null}
		\opciones{$x(x-3)$}
				{$x-3$}
				{$x$}
				{$x(x-1)$}
				
	\pregunta{La solucion de la ecuación
			$\dfrac{x}{3}-\dfrac{x-5}{4}=2$ es}
			{null}
		\opciones{8}
				{9}
				{10}
				{11}
				{12}
	\pregunta{Las raices de la ecuacuión
			$4x^2+16x+15=0$ son}
			{null}
		\opciones{$x_{1}=\dfrac{-3}{4};-----x_{2}=-5$}
				{$x_{1}=\dfrac{-3}{2};----- x_{2}=\dfrac{-5}{2}$}
				{$x_{1}=3; ------x_{2}=\dfrac{-5}{4}$}
				{$x_{1}=\dfrac{3}{2};------ x_{2}=\dfrac{5}{2}$}
				{$x_{1}=\dfrac{-3}{4};------ x_{2}=5$}
				
	\pregunta{Simplificando 
			$\dfrac{x^2-5x+6}{2ax-6a}$ se obtiene}
			{null}
		\opciones{$\dfrac{(x-2)(x-3)}{2a}$}
				{$\dfrac{(x-3}{2a(x-2}$}
				{$\dfrac{x+2}{2a}$}
				{$\dfrac{x-2}{2a}$}
				{$\dfrac{(x-2)^2}{2a}$}	
	\pregunta{Al resolver la desigualdad
			$-2x+6>=16$ se obtiene}
			{null}
		\opciones{$x>5$}
				{$x<-5$}
				{$x>=5$}
				{$x<=-5$}
				{$x=5$}		
	\pregunta{La solución al siguiente sistema de euaciones es}{img/sistema}
			
		\opciones{$x=-2,y=3,z=5$}
				{$x=2,y=4,z=5$}
				{$x=-2,y=4,z=5$}
				{$x=2,y=3,z=-5$}
	\pregunta{Si $f(x)=\dfrac{1}{x^2-1}$ y f(x)=x+2 entonces (f o g)(x) es igual}
			{null}
		\opciones{$\dfrac{1}{x^2+2}$}
				{$\dfrac{1}{x^2+3}$}
				{$\dfrac{1}{x^2+5}$}
				{$\dfrac{1}{x^2+4x+3}$}
				
	\pregunta{Determina el valor de sen(B) en el triángulo mostrado.}
			{img/triangulo}
		\opciones{$\dfrac{2}{\sqrt{5}}$}
				{$\dfrac{1}{\sqrt{5}}$}
				{$\dfrac{1}{\sqrt{3}}$}
				{$\dfrac{2}{\sqrt{3}}$}
				
	\pregunta{Determina el dominio de la función y=log(x-2.}
			{null}
		\opciones{$(2,\infty)$}
				{$[2,\infty)$}
				{$(-\infty,2)$}
				{$(-\infty,2])$}
				
				
	\pregunta{La distancia entre los puntos A=(-7,-2) y B=(2,7) es}
			{null}
		\opciones{$10$}
				{$\sqrt{50}$}
				{$\sqrt{162}$}
				{$50$}
				
	\pregunta{Ecuacion que corresponde a una circunferencia de radio = $\sqrt{2}$ y centro en C=(0,0).}
			{null}
		\opciones{$x^2-y^2=2\sqrt{2}$}
				{$x^2+y^2=\sqrt{2}$}
				{$x^2+y^2=4$}
				{$x^2+y^2=2$}
				
	\pregunta{Calcula el vértice de la parábola $(y-3)^2 = 12x-24$}
			{null}
		\opciones{$V=(2,3)$}
				{$V=(3,2)$}
				{$V=(-3,-2)$}
				{$V=(-2,-3)$}
	\pregunta{El rango de la función cuya gráfica se muestra a continuación, son los valores}{img/grafica}
		\opciones{mayores que 0 y menores que 1}
				{menores que 0}
				{mayores que 2}
				{mayores o iguales que 0}
				{maneores o iguales que 0}
	\pregunta{$\dfrac{\pi}{3}$ radianes es igual}
			{null}
		\opciones{30°}
				{45°}
				{60°}
				{90°}
				{180°}
	\pregunta{La pendiente de la recta 3x+6y-1 = 0 es}
			{null}
		\opciones{-6}
				{-3}
				{$\dfrac{-1}{2}$}
				{$\dfrac{1}{2}$}
				{3}
	\pregunta{La ordenada de la recta 6x+4y-4 = 0 es}
			{null}
		\opciones{(1,1)}
				{(0,1)}
				{(-4,1)}
				{(1,0)}
				{(0,4)}
				
	\pregunta{¿Cuál es la ecuación de una elipse horizontal con centro en (2,1)}
			{null}
		\opciones{$\dfrac{(x-2)^2}{16}+\dfrac{(y-1)^2}{16}=1$}
				{$\dfrac{(x-2)^2}{64}+\dfrac{(y-1)^2}{144}=1$}
				{$\dfrac{(x-2)^2}{144}+\dfrac{(y-1)^2}{64}=1$}
				{$\dfrac{(x+2)^2}{144}+\dfrac{(y+1)^2}{64}=1$}
				{$\dfrac{(x+2)^2}{16}+\dfrac{(y+1)^2}{25}=1$}
	\pregunta{En terminos de sen(a) y cos(a), tan(a) es igual a}
			{null}
		\opciones{$\dfrac{sen(a)}{cos(a)}$}
				{$\dfrac{cos(a)}{sen(a)}$}
				{$\dfrac{2sen(a)}{cos(a)}$}
				{$\dfrac{2cos(a)}{sen(a)}$}
				{$\dfrac{2cos(a)}{2sen(a)}$}
	\pregunta{El valor del limite $\lim\limits_{x \to 1}\dfrac{x^2-x}{x}$ es igual a}
			{null}
		\opciones{-2}
				{-1}
				{0}
				{1}
				{2}
				
				
	\pregunta {La derivada segunda derivada de $y= 3x^4+5x^2+3$ es } 
	{null}
		\opciones
			{$y''=12x^3+10x $}
			{$y''= 36x^2+10$}
			{$y'=36x^2+10$}
			{$y''= 72x$}		
	\pregunta {La derivada de y=ln($x^2+5$) es}{null}
		\opciones
			{$\dfrac{1}{x^2+5}$}
			{$\dfrac{2x}{x+5}$}
			{$\dfrac{2x}{x^2+5}$}
			{$\dfrac{2x}{x^2-5}$}
	\pregunta {El valor de la siguiente integral $\int x^2+6 \,dx$}
		{null}
		\opciones{$\dfrac{x^3}{3}+6+c$}
			{$\dfrac{1}{3}x^3+6x+c$}
			{$\dfrac{1}{3}x^3+6x+$}
			{$\dfrac{3}{2}x^3+6x+c$}
			
	\section{Física}
	
	\pregunta{Un auto arranca con una aceleración
constante de 18 m/$s^2$
; la velocidad del
auto dos segundos después de iniciar
su movimiento es de}
			{null}
		\opciones{9 m/s}
				{18 m/s}
				{32 m/s}
				{36 m/s}
		\pregunta{Un sistema esta en equilibrio térmico cuando}
			{null}
		\opciones{en un proceso su temperatura no
varía}
				{su temperatura es igual a la de otro
sistema con el que está en contacto
térmico}
				{sus propiedades termodinámicas no
cambian.}
				{su volumen y su presión permanecen
constantes.}

	\pregunta{De acuerdo a la imagen, la corriente
que circula por el resistor de
resistencia
$\dfrac{R}{2}$ es . (Considera V= RxI)}
			{img/diagrama}
		\opciones{$\dfrac{V}{2R}$}
				{$\dfrac{2V}{R}$}
				{$\dfrac{V}{3R}$}
				{$\dfrac{3V}{R}$}
		\pregunta{La presión atmosférica en el Everest
disminuye comparada con la del nivel
del mar porque}
			{null}
		\opciones{La densidad el aire cambia.}
				{La altura de la capa de aire soportada es menor}
				{La presión hidroestática del mar fluye}
				{La densidad del aire soportada es mayor.}
				
		\pregunta{Actualmente se concibe a la luz con
un comportamiento dual, esto se
refiere a que se le considera como}
			{null}
		\opciones{partícula y onda}
				{energía y movimiento}
				{calor y difracción}
				{reflexión y refracción}
				
		\pregunta{Sobre un cuerpo se aplicaron diferentes fuerzas en dirección horizontal y con el mismo sentido, provocando que el objeto experimentara distintas aceleraciones, Basándote en la gráfica de los resultados y despreciando la accion ejercidad por la fricción entre el objeto y la superficie de contacto ¿cuál es la masa del objeto?.(considera F= ma)}
			{img/masa}
		\opciones{1.6 kg}
				{2.5 kg}
				{3.2 kg}
				{4.0 kg}
				{6.4 kg}
				
		\pregunta{La ecuación que te permite calcular la energía cinética de una partícula de masa m y velocidad v es }
			{null}
		\opciones{Ec = mv}
				{Ec = $2m^2$}
				{Ec = $\dfrac{1}{2}mv^2$}
				{Ec = m$(\dfrac{v}{2})^2$}
		\pregunta{La teoría cinética de los gase predice una temperatura para la cual todas la partículas del gas cesan su movimiento aleatoria. ¿Cuál es esa temperatura?}
			{null}
		\opciones{-100 °C}
				{0 °F}
				{0 °K}
				{0 °R}
				{0 °C}
			\pregunta{Un carrito del supermecado golpea el tobillo de una señora. ¿Cuál de los siguiente enunciados es verdadero?}
			{null}
		\opciones{Solamente está presente la fuerza que ejerce el carrito sobre el tobillo}
				{Los dos, carrito y tobillo, reciben una fuerza pero la ejercida por el carrito es de menor magnitud}
				{Solamente está prssente la fuerza que ejerce el tobillo sobre el carrtio}
				{Las dos, carrito y tobillo, reciben una fuerza de igual magnitud}
				{Las dos, carrtio y tobillo, reciben una fuerza pero la ejercidad por el tobillo es de menor magnitud.}
			\pregunta{Si se deja caer una piedra desde la misma altura sobre la superfice luna y sobre la superficie terrestre, ¿qué afirmación es correcta?}
			{null}
		\opciones{En los dos casos la piedra tarda el mismo tiempo ya que se está dejando caer desde la misma altura }
				{En la luna la piedra se queda flotando a la altura donde se soltó porque no hay atmósfera}
				{En la luna tarda más tiempo en llegar al suelo que en la tierra, porque en la Luna disminuye la densidad de la piedra}
				{En la Luna la piedra cae más rápido porque no hay fricción debido a la atmósfera.}
				{En la luna tarda más tiempo en llegar al suelo que en la tierra, porque en la Luna es menor la fuerza de gravedad.}
				
		\pregunta{Para medir fuerzas se emplea un dinamómetro que en escencia es un resorte adecuadamente calibrado. En la calibración de todo dinamómetro se hace uso de}
			{null}
		\opciones{la ley de la gravitacion universal.}
				{la segunda ley de Newton.}
				{el principio de la conservación de la masa.}
				{ley de Hooke.}
				{el principio de conservación de movimiento.}
		\section{ESPAÑOL}
\pregunta{Elija la opción que contenga el tipo de discurso de acuerdo al fragmento.\\
Me llamo Jean Valjean. Soy prisionero. He pasado diecinueve años en la cárcel. Estoy libre desde hace cuatro días y me dirijo a Pontarlier, que es mi destino}{null}
\opciones{narrativo}{argumentativo}{descriptivo}{informativo}
\pregunta{Sabe que su hijo, educado desde su más tierna infancia en el hábito y la precaución del peligro, puede manejar un fusil y cazar. Aunque es muy alto para su edad, no tiene sino trece años. Y parecería tener menos, a juzgar  por la pureza de sus ojos azules, frescos aún de sorpresa infantil.}{null}
\opciones{narrativo}{argumentativo}{descriptivo}{informativo}
\pregunta{La cabeza te da vueltas, inundada por el ritmo de ese vals lejano que suple la vista, el tacto, el olor de plantas húmedas y perfumadas: caes agotado sobre la cama, te tocas los pómulos. Los ojos, la nariz, como si temieras que una mano invisible te hubiese arrancado la máscara has llevado durante veintisiete años.}{null}
\opciones{argumentativo}{descriptivo}{narrativo}{informativo}
\pregunta{Es importante, entonces, reflexionar acerca de la enorme importancia que tiene el hacer buen uso de la lengua y, precisamente, la lectura y la redacción constituyen las herramientas que nos ampliarán las puertas del universo de las letras.}{null}
\opciones{descriptivo}{argumentativo}{narrativo}{informativo}
\pregunta{Don Paco sigue gozando de la privanza del cacique y gobernando en su nombre cuanto hay que gobernar en la villa. Juanita, casada con él, lo adora, lo mima  y le ha dado dos hermosísimos pimpollos.}{null}
\opciones{argumentativo}{narrativo}{descriptivo}{informativo}
\pregunta{Lea el siguiente texto y responda a las siguientes preguntas:\\
Psicología del Vestido\\
Voy a hablar de un tema banal en apariencia, la psicología del “vestido”. Quisiera alcanzar a demostraros que no lo es tanto como parece. Su interés es en efecto excepcional, porque el vestido constituye una necesidad primaria prácticamente del mismo rango que el alimento. El pan y el traje se consideran como dos necesidades reales e igualmente perentorias. El mismo sentimiento de injusticia nos produce el espectáculo de un hombre que no tiene nada de comer y el de otro que está completamente desnudo; y, sin embargo, hay entre ambos hechos una diferencia esencial. El comer es una esclavitud con la que hemos nacido, que nos iguala a todos los demás seres de la tierra, mientras que el vestido es una creación artificial exclusiva de la especie humana. ¿Qué relación tiene, nos preguntamos, entonces, el traje con los instintos primarios para haberse convertido en una de las características de nuestra raza? Es evidente que el hombre, en la aurora de su vida sobre el planeta, estuvo largo tiempo desnudo; su piel recia y el vello abundante que la cubría, era suficiente para defenderlo del rigor del ambiente.\\
Dr. Alexis Carre\\
¿Cuál es la idea central del texto?}{null}
\opciones{La moda}{El buen vestir}{La necesidad del vestido}{El alimento}
\pregunta{El autor al hacer una comparación entre el comer y el vestir ¿en qué difieren estos aspectos?}{null}
\opciones{El vestir es algo primitivo}{La comida es algo secundario}{El comer es exclusivo del ser humano}{El vestir es una creación artificial}
\pregunta{¿Qué significado tiene la palabra perentorias en el texto?}{null}
\opciones{concluyentes}{pasivos}{apremiantes}{perennes}
\pregunta{Usted diría que el vestido:}{null}
\opciones{No es una necesidad secundaria}{Es una necesidad terciaria}{Es la más importante de las necesidades}{Es más importante que la comida}
\pregunta{¿Según el texto, qué relación tiene el traje con los instintos primarios del hombre?}{null}
\opciones{Supervivencia}{Abastecimiento}{Protección}{Jerarquía}
\pregunta{Identifique que tipo de oraciones se presentan a continuación:\\
El aire era pesado allí adentro}{null}
\opciones{simple}{subordinada}{compuesta}
\pregunta{Los faroles se apagaron muy tarde}{null}
\opciones{compuesta}{subordinada}{simple}
\pregunta{Es importante que estudies}{null}
\opciones{simple}{compuesta}{subordinada}
\pregunta{La curva que aparece a la vista es muy peligrosa}{null}
\opciones{subordinada}{simple}{compuesta}
\pregunta{Nadie me ayudó nunca en nada}{null}
\opciones{simple}{compuesta}{subordinada}
\pregunta{Identifique qué tipo de sujeto se encuentra en las siguientes oraciones:}{null}
\opciones{expreso}{tácito}{nominal}{verbal}
\pregunta{Son abundantes y majestuosos los manantiales que corren en mi pueblo.}{null}
\opciones{expreso}{tácito}{nominal}{verbal}
\pregunta{Sin embargo, nos proporcionó toda la información}{null}
\opciones{nominal}{tácito}{verbal}{expreso}
\section{GEOGRAFÍA}
\pregunta{Divide a la Tierra en los hemisferios Occidental y Oriental.}{null}
\opciones{El Ecuador.}{El meridiano cero.}{El Trópico de Cáncer.}{El Trópico de Capricornio.}{El eje terrestre.}
\pregunta{Es la teoría que considera que la corteza terrestre está constituida por bloques de roca de gran tamaño.}{null}
\opciones{Deriva continental.}{Tectónica de placas.}{Del origen continental.}{Epirogénica.}{Orogénica.}
\pregunta{El conjunto de movimientos que afectan la corteza terrestre y provocan que las capas rocosas se deformen, rompan y reacomoden se llama}{null}
\opciones{fracturas.}{fallas.}{vulcanismo.}{plegamiento.}{tectonismo.}
\pregunta{Es la coordenada que indica la distancia angular que hay entre un punto de la superficie terrestre y el meridiano de Greenwich.}{null}
\opciones{Longitud.}{Altitud.}{Latitud.}{Latitud alta.}{Latitud media.}
\pregunta{¿Cuál es el océano con mayor extensión en el mundo?}{null}
\opciones{Atlántico.}{Glacial Ártico.}{Pacífico.}{Índico.}{Ártico.}
\pregunta{Selecciona los movimientos de los océanos.\\
a. Corrientes marinas.\\
b. Mares.\\
c. Canales.\\
d. Olas.\\
e. Mareas.}{null}
\opciones{a, b, c.}{a, c, d. }{b, d, e.}{c, d, e.}{a, d, e.}
\pregunta{Identifica los factores que intervienen en el modelo del relieve continental.\\
a. Vulcanismo.\\
b. Tectonismo.\\
c. La depositacion de sedimentos acarreados por los ríos\\
d. Agentes de erosión, como el agua y el viento.\\
e. Las cadenas montañosas.}{null}
\opciones{a y b}{b y c.}{b y d.}{c y d.}{d y e.}
\pregunta{¿Qué elemento no pertenece al conjunto?\\
1.	Asia\\
2.	África\\
3.	Europa\\
4.	Australia\\
5.	Oceanía\\
6.	América}{null}
\opciones{Asia.}{África.}{Oceanía.}{Australia.}{América.}
\pregunta{¿Cuál es el nombre de los elementos que se incluyen en el reactivo anterior?}{null}
\opciones{Los territorios de mayor latitud.}{Los países con menos longitud.}{Los territorios que carecen de población permanente.}{Los continentes.}{Las regiones del mundo.}
\pregunta{Los aspectos que permiten agrupar a los países en desarrollo y en vías de desarrollo son\\
a.el desarrollo industrial y el económico.\\
b.la transformación de los recursos naturales.\\
c.nivel de vida de la población.\\
d.el financiamiento de transporte.\\
e. las exportaciones e importaciones de las actividades económicas.}{null}
\opciones{a, b, d.}{b, d, e.}{d, e, c.}{a, c, e.}{a, b, e.}
\section{HISTORIA DE MÉXICO}
\pregunta{Ley del 25 de junio de 1856 para desamortizar los bienes del clero y suprimir toda forma de propiedad comunal.}{null}
\opciones{Ley lerdo}{Ley Iglesias}{Ley Juárez}{Ley Ocampo}
\pregunta{Cuando el presidente Juárez regresó a la capital en 1861, una de sus medidas fue:}{null}
\opciones{Dictar las Leyes de Reforma}{Declarar la guerra a Estados Unidos}{Perdonar a los desertores}{Suspender el pago de la deuda exterior}
\pregunta{Durante la administración del presidente Juárez se realizaron reformas educativas importantes como la:}{null}
\opciones{Creación del Museo de Historia}{Apertura de la Universidad}{Creación de la Escuela Nacional Preparatoria} {Fundación del instituto de Geología}
\pregunta{Entre los objetivos de la Revolución de Ayutla destaco la.}{null}
\opciones{Creación del Segundo}{Imperio}{Libertad de cultos}{Destitución del presidente Santa Anna}{Reinstalación de Mariano Arista en la presidencia}
\pregunta{Las Leyes de Reforma son importantes en la Historia de México porque:}{null}
\opciones{Fueron impulsadas por el partido moderado}{Establecieron las bases de un desarrollo económico}{Abolieron los privilegios de los militares y religiosos}{Limitaron la participación política de los extranjeros}
\pregunta{Porfirio Díaz llegó al poder en 1876 mediante el Plan de:}{null}
\opciones{Xochimilco}{Agua Prieta}{Tuxtepec}{Casa Mata}
\pregunta{La filosofía política que sustentó el gobierno de Porfirio Díaz fue el:}{null}
\opciones{Romanticismo}{Materialismo}{Idealismo{Positivismo}
\pregunta{La economía durante el Porfiriato se caracterizó porque estuvo:}{null}
\opciones{Dominada por extranjeros}{Dedicada exclusivamente a la exportación}{Reducida a las importaciones}{Impulsada por u alto nivel económico}
\pregunta{Grupo político que ejerció gran influencia en la política porfirista.}{null}
\opciones{Magonistas}{Científicos}	{Juaristas}{Lerdistas}
\pregunta{Nombre del Plan con el cual Francisco I. Madero enarboló el lema Sufragio efectivo, no reelección}{null}
\opciones{San Luís}{Ayala}{Guadalupe}{Noria}
\pregunta{La principal demanda del plan de Ayala promulgada por Emiliano zapata en 1911 fue:}{null}
\opciones{fijar un horario de ocho horas de trabajo}{restituir la tierra a los pueblos}{privatizar los bosques y aguas}{crear reservas y parques nacionales}
\section{HISTORIA UNIVERSAL}
\pregunta{Son pensadores del movimiento iluminista del siglo XVIII, los siguientes excepto:}{null}
\opciones{John Locke.}{J. J. Rousseau.}{Denis Diderot.}{Roberto Turgot.}
\pregunta{A finales del siglo XVIII el desarrollo de la generación de vapor aceleró:}{null}
\opciones{La Revolución} Industrial.}{La Revolución} Soviética.{El Liberalismo.}{La Revolución Francesa.}
\pregunta{Son ideas características del movimiento del Siglo de las Luces.\\
a.	Los seres humanos nacen libres e iguales en derechos.\\
b.	Las mujeres deben participar en política.\\
c.	El pueblo le debe obediencia ciega al monarca.\\
d.	Es necesario que el poder se divida en tres: Legislativo, Ejecutivo y Judicial.\\
e.	La soberanía reside en el pueblo.}{null}
\opciones{a, b, y c.}{b, d y e.}	{a, d y e}{b, c y d.}
\pregunta{¿Qué nombre se le dio a la religión natural racionalista en la ilustración?}{null}
\opciones{Marxismo}{Deísmo}{Liberalismo}{Existencialismo}{Socialismo}
\pregunta{Filósofo que apoyo la monarquía mediante un contrato social:}{null}
\opciones{Locke}{Montesquieu}{Rousseau}{Hobbes}{Voltaire}
\pregunta{Expreso que la soberanía nacional reside en el pueblo}{null}
\opciones{Locke}{Voltaire}{Hobbes}{Rousseau}{Montesquieu}
\pregunta{Pensador ingles que ánima a la población a empuñar las armas para derrocar al mal gobierno}{null}
\opciones{Locke}{Montesquieu}{Adam smith}{David Ricardo}{Emmanuel Kant}
\pregunta{¿Que pensador considera al régimen parlamentario inglés como el sistema optimo para conservar el equilibrio político}{null}
\opciones{Montesquieu}{David Hume}	{Robespierre}{Voltaire}{Rosseau}
\pregunta{El centro de desarrollo en la época Ilustrada fue en:}{null}
\opciones{Inglaterra}{Francia}{España}{Italia}{Austria}
\pregunta{¿Quiénes llevaron la dirección del movimiento Enciclopedista?}{null}
\opciones{Montesquieu y Hobbes}{Hobbes y Locke}{Diderot y Rosseau}{Voltaire y Descartes}{Diderot y D.Lambert}
\pregunta{Newton, Descartes, Smith, Lavoisier, destacaron en los campos de la:}{null}
\opciones{Física, Filosofía, Economía y Química}{Medicina, Matemáticas, Astronomía y Humanidades}{Medicina, Filosofía, Biología y Química}{Física, Arqueología, Economía y Geografía}{Economía, Física, Filosofía y Matemáticas	}		


\section{Bilogía}
	
	\pregunta{Investigador que le asigno el nombre a la celula}{null}
	\opciones{Rudolf Virchow}
		{Robert Brown}
		{Pasteur}
		{Robert Hooke}	
	% A. Robert Hook 
	
	\pregunta{Organelo cuya funcion es transportar moleculas intracelularmente}
	{null}
	\opciones{Membrana y ncuelo}
		{Mitocondrias y nucleo}
		{Ribosomas y cloroplastos}
		{Aparato de Golgi y reticulo endoplasmatico}
		
	\pregunta{En la fosotintesis la reaccion no dependiente de luz se lleva acabo en:}{null}
	\opciones{Mitocondrias}
		{Tilacoides}
		{Grana} 
		{Estroma}
		
	\pregunta{Producto final del glucolisis}{null}
	\opciones{2 ATP, 2 NADH, 2 Piruvatos}
		{3 ATP, 3 NADH, 3 Piruvatos}
		{6 ATP, 2 FADH, 2 Piruvatos}
		{2 ATP, 2 FADH, 1 Piruvato}

	\pregunta{en la actualidad es la teoría mas aceptada sobre el origen de los sere vivos}{null}
	\opciones{Creacionista}
	{Endosimbiótica}
	{Vitalismo}
	{Quimiosintética}

	\pregunta{Cual es el orden de fases de la mitosis}{null}
	\opciones{Profase, Metafase, Telofase, Anafase}
		{Anafase, Profase, Metafase, Telofase}
		{Profase, Telofase, Metafase, Anafase}
		{Metafase, Anafase, Telofase, Profase}
		
	\pregunta{Numero de cromosomas que posen los gametos y las células somáticas}{null}
	\opciones{22 y 38}
		{23 y 36}
		{24 y 37}
		{36 y 23}
		
	\pregunta{El sexo cromosómico se establece durante el proceso de}{null}
	\opciones{Ovulación}
		{Segmentación}
		{Implantación}
		{Fecundación}
		
	\pregunta{Una célula somática presenta un número cromosómico con 46 XX; durante la mitosis, al dividirse, es de esperarse que forme células con }{null}
	\opciones{23 X}
		{23 XX.}
		{46 XX.}
		{46 XY.}
		
	\pregunta{Relaciona las fases del ciclo celular con los procesos que ocurren en cada una de ellas.\\
Fases\\
	I. Mitosis.\\
	II. Interfase.\\
	Procesos\\
	a. Duplicación del ADN.\\
	b. Crecimiento de la célula.\\
	c. Síntesis de proteínas.\\
	d. División celular.}{null}
	\opciones{I: a – II: b, c, d}
		{I: d – II: a, b, c}
		{I: c, d – II: a, b}
		{I: b, c – II: a, d}
		
	\section{QUIMICA}

	\pregunta{¿Cuál es la molaridad de una disolución que contiene 20g de NaOH en 2L de solución?\\
	Masa atómica:  Na: 23	O: 16	H:1}{null}
	\opciones{0.25M}
		{0.50M}
		{0.75M}
		{1.00M}

	\pregunta{¿En cuál de las siguientes opciones hay materiales formados únicamente por elementos?}{null}
	\opciones{Na(g), Cl2(g), P4(s)}
		{O2(g), He(g), CO(g)}
		{S8(g), N2(g), SO2(g)}
		{CO(g), Na(s), S8(s)}

	\pregunta{¿Cuál reacción representa la formación de una sal?}{null}
	\opciones{SO2 + H2O → H2SO3}
		{Cl2 + H2O → HCl}
		{N2O5 + H2O → 2HNO3}
		{HCl + KOH →KCl + H2O}

	\pregunta{¿Cuál de los siguientes valores de pH corresponde a la mayor concentración de iones OH–?}{null}
	\opciones{2}
		{7}
		{8}
		{13}
	\pregunta{Relaciona los compuestos con la función que les corresponde.\\
Compuestos\\
	 I. LiOH\\
	II. H3PO4\\
	III.NaH \\
	Funciones\\
	a. Ácido.\\
	b. Hidróxido.\\
	c. Sal. }{null}
	\opciones{I:c – II:b –III:a}
		{I:b – II:c –III:a}
		{I:b – II:a – III:c}
		{I:a – II:c –III:b}

	\pregunta{¿Cuál es el enunciado verdadero? }{null}
	\opciones{El aire es un compuesto y el cloruro de sodio es una mezcla.}
		{El cloruro de sodio es un elemento y la plata es un compuesto.} 
		{El aire es una mezcla y la plata es un compuesto. }
		{El cloruro de sodio es un compuesto y el aire es una mezcla.}

	\pregunta{Al enlace que une a las moléculas de agua se le denomina }{null}
	\opciones{covalente. }	
		{iónico. }
		{coordinado. }
		{puente de hidrógeno.}

	\pregunta{En el aire que respiramos, el elemento gaseoso que se encuentra en mayor cantidad es el }{null}	

	\opciones{hidrógeno. }
		{nitrógeno. }
		{oxígeno. }
		{ozono}

	\pregunta{La función principal de un catalizador es favorecer que }{null}
	\opciones{aumente la cantidad de reactivos sin reaccionar.}
		{los productos tengan mayor pureza. }
		{los reactivos se consuman más rápido.}
		{aumente la temperatura de los reactivos.}

	\pregunta{Una proteína está formada por }{null}
	\opciones{una serie de enzimas. }
		{una cadena de aminoácidos. }
		{un polímero de carbohidratos. }
		{un conjunto de triglicéridos.}
		
	\section{Literatura}
	
	\pregunta{Elige la función de la lengua que predomina en el siguiente ejemplo. Luisa, ¿puedes limpiar la mesa y lavar los trastes por favor? }{null}
	\opciones{Metalingüística.}	
		{Apelativa.}
		{Referencial.}
		{Sintomática.}
		
	\pregunta{Identifica el enunciado en el que la lengua está usada en su función poética. }{null}
		\opciones{Era apenas una niña cuando la vi por primera vez.}
			{A las tres en punto moriría un transeúnte.}
			{Las nieves del tiempo platearon mi sien. }
			{Chopin soñó que estaba muerto en el lago}
			
	\pregunta{¿Qué modo discursivo predomina en el siguiente párrafo?
 El alcoholismo es una enfermedad progresiva y crónica, que presenta síntomas que van desde el malestar hasta el dolor intenso. Depende de varios factores, principalmente de la predisposición genética y de la influencia del medio ambiente familiar y social. Pese a que afecta todo el cuerpo y provoca una variedad de problemas médicos, los principales síntomas se manifiestan en el sistema nervioso. A través de éste, en especial del cerebro, la adicción produce diversos trastornos en el pensamiento, las emociones y la conducta del enfermo. }{null}
 		\opciones{Instrucción.}
 			{Descripción. }
 			{Enumeración. }
 			{Explicación.}
 			
 	\pregunta{¿Qué modo discursivo predomina en el siguiente ejemplo? 
El libro comprende tres capítulos, con cinco subtemas cada uno. Sin embargo, no tiene consistencia. Esto se corrobora, en primer lugar, porque carece de un apartado de conclusiones. En segundo lugar, no cita las fuentes bibliográficas en las que se apoya. Esto hace que el texto sea de poco fiar. }{null}
		\opciones{Enumeración. }
			{Descripción. }
			{Argumentación.}
			{Narración.}
			
	\pregunta{¿En qué versos del siguiente poema de Sor Juana Inés de la Cruz aparece una metáfora? \\
Al que ingrato me deja, busco amante; 1 \\
 al que amante me sigue, dejo ingrata; 2 \\
constante adoro a quien mi amor maltrata; 3 \\
 maltrato a quien mi amor busca constante. 4 \\
Al que trato de amor, hallo diamante, 5 \\
y soy diamante al que de amor me trata; 6 \\
triunfante quiero ver al que me mata, 7 \\
y mato al que me quiere ver triunfante. 8 \\
Si a éste pago, padece mi deseo; 9 \\
si ruego a aquél, mi pundonor enojo: 10 \\
de entre ambos modos infeliz me veo. 11 \\
Pero yo, por mejor partido escojo 12 \\
de quien no quiero, ser violento empleo, 13 \\
que, de quien no me quiere, vil despojo. 14  \\}{null}
		\opciones{1, 3 y 7 }
			{5 y 6 }
			{7 y 14 }
			{1 y 2}
	\pregunta{Elige las características del poema lírico. }{null}
	\opciones{Objetividad, profundidad y extensión. }
		{Argumentación, objetividad y ejemplificación. }
		{Individualismo y subjetividad. }
		{Veracidad, exactitud y desenlace}
	
	\pregunta{Poeta mexicano de la segunda mitad del siglo XX, ganador del premio Nobel. }{null}
	\opciones{Carlos Fuentes. }
		{Jaime Sabines. }
		{Octavio Paz. }
		{Carlos Monsiváis.}
	
	\pregunta{Un cuento se diferencia de una novela porque éste tiene }{null}
	\opciones{amplio desarrollo psicológico de los personajes.}
		{desarrollo elaborado y rápido desenlace. }
		{brevedad y rápido desenlace. }
		{intensidad y múltiples hilos narrativos}
			
	\pregunta{Movimiento literario que surge en la segunda mitad del siglo XIX, como reacción ante el individualismo extremo y el idealismo que caracterizó al Romanticismo.}{null}
	\opciones{Neoclasicismo. }
		{Vanguardismo. }
		{Realismo. }
		{Surrealismo}



				
				
				
				


	
				
				
				
	
					
				
				
				
				
	\end{multicols*}

\end{document}