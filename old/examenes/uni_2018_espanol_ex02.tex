\documentclass[10pt,a4paper]{article}
	\usepackage{../miconfiguracion}
		\configPage
\begin{document}
	\BgThispage
	\makemytitle{Español}{Primer examen}
	\datosalu
	%\marcaagua	
	%SECCION DE PREGUNTAS
	\begin{multicols*}{2}
\pregunta{Selecciona la opción que presenta un enunciado con predominio de la función poética de la lengua.}{null}
\opciones{Ese verso muestra su vocación poeta}{La literatura incluye la poesía}{La poesía emociona al sensible.}
{Las golondrinas firman el pergamino del cielo}
\pregunta{Identifica el enunciado en el que la lengua está usada en su función poética.}{null}
\opciones{Era apenas una niña cuando la vi por primera vez}{A las tres en punto moriría un transeúnte.}{El convento era espantosamente triste.}{Las nieves del tiempo platearon mi sien.}
\pregunta{A partir de lo que se implica o se afirma en el texto siguiente, contesta las preguntas 3 a 6.\\
LOS RECUERDOS\\
El siglo se resiste a morir y en sus estertores descarga grandes manotazos sobre el oriente y el occidente. El hundimiento de las bolsas mundiales es uno de los espasmos con que el siglo pretende paralizar el tiempo, ahuyentar su fin, contener la evolución de las cosas.\\
Mientras los ochenta fueron años de las movidas que auguraban el germen de un espacio nuevo, los noventa se presentan compactos y recurrentes. Sobre la superficie de este decenio la historia patina y se rebobina a si misma. El destino de este tiempo era cumplir con la ratificación del milenio, pero como víctima de un miedo resbaladizo, cada paso adelante ha sido a menudo un \textit{revival.}Más que abrirse a los desconocido,la historia de los noventa ha consistido, sobre todo, en un desfile del repertorio ya emitido. Se ha portado así este tramo como el capítulo último de muchos programas de televisión que justo el día de sus despedida ofrecen los fragmentos más famosos de sus producción. Desde el regreso de los nacionalismos al miedo al sexo; desde las guerras nucleares y raciales a las mdas étnicas o el gusto por lo exótico; desde la utopía de la naturaleza o la fascinación por los robots hasta el retorno de la irracionalidad y las emociones, el miedo a la ciencia, el aumento de las desigualdades sociales, la presencia de nuevas plagas y hambrunas, el arte de la provocación, o los \textit{reviva}en el teatro, el mobiliario, el maquillaje, el diseño de automóviles, los vestidos, las películas, la historia se copia a si mismas.\\
Sobre el desagüe del siglo XX se ha emplazado un tapón donde las energías rebotan en forma de fanatismos, recesiones económicas y parodias de los años vividos.\\
Como el ser que, llegado el ocaso de su existencia, hace vivir una reminiscencia del pretérito, el siglo XX se complace ahora en su detenida evocación de anciano, a despecho de ls apremios del progreso.\\\\
Entre las décadas de los ochenta y los noventa existe una relación de}{null}
\opciones{complementación.}{integración.}{continuación.}{oposición.}
\pregunta{El hundimiento de las bolsas de valores representa}{null}
\opciones{el inicio de un nuevo orden mundial.}{un efecto de cambios promovidos en los ochenta.}{un recurso del siglo XX para aplazar el cambio.}{Un hecho recurrente en el sistema financiero internacional.}
\pregunta{En la década de los noventa}{null}
\opciones{son importantes los programas de televisión.}{se ha dado un progreso espectacular.}{se ha dado el retorno hacia la irracionalidad.}{han disminuido las desigualdades sociales.}
\pregunta{La comparación entre los noventa y el último capítulo de un programa de televisión se debe a que ambos}{null}
\opciones{contienen la evolución del progreso.}{auguran el germen de un espacio nuevo.}{cumplen la ratificación del milenio.}{repiten los fragmentos más reconocidos.}
\pregunta{Identifica el enunciado que representa un sujeto TÁCITO o IMPLÍCITO.}{null}
\opciones{Somos nada frente a la muerte infausta.}{¡Señor!, tiembla mi alma ante tu grandeza.}{Yo he subido más alto, mucho más alto.}
\pregunta{Identifica el enunciado en el cual existe un error sintáctico que provoca distintas interpretaciones.}{null}
\opciones{Con asombro, Juan vio a Martín engañando a su suegra.}{El defensa reclamó una falta al árbitro que nunca existió}{Un día que salió a da una vuelta encontró a su chofer.}{Pensaba que lo disculparían por todo lo malo que hiciera.}
\pregunta{Elige el enunciado en el cual existe un error sintáctico que provoca distintas interpretaciones.}{null}
\opciones{Los checoslovacos combatían en las calles y se oponían a la dictadura.}{Un sinnúmero de feligreses acudió a la Villa.}{La creación de muchas cosas no se tenían contempladas.}{El constante flujo y reflujo de divisas provocó alarma.}
\pregunta{De acuerdo con la relación de ANALOGÍA entre las palabras en letras mayúsculas, señala la opción que presenta una relación semejante en la pregunta siguiente.\\ BIEN.VALOR.}{null}
\opciones{maldad.bondad}{guía.pecado.}{verdad.escándalo.}{poder,orgullo.}
\pregunta{Según el sentido del siguiente enunciado elige la opción que es SINÓNIMO de la palabra en letras mayúsuculas.\\
Nunca ha dejado de causar extrañeza que los hombres homéricos en el dolor más PROFUNDO pueden consolarse con saber que su destino resonará en los cantos del futuro.}{null}
\opciones{Grave.}{Intenso.}{Abismal.}{Bajo.}
\pregunta{En las siguientes tres preguntas, elige el enunciado que tiene la ortografía CORRECTA.}{null}
\opciones{El rebaño avanza sin cesar y ellos comienzan a rezagarse.}{El rebaño avanza sin cesar y ellos comiensan a rezagarze.}{El rebaño avanza sin cezar y ellos comienzan a rezagarce.}{El rebaño avanza sin cesar y ellos comienzan resargase.}
\pregunta{Enunciado con ortografía CORRECTA.}{null}
\opciones{Notó que auqel beso era de un extraño.}{Los animales se detenían allí sin contemplarlo.}{Pronto adquirio la costumbre de desvelarse.}{El muchacho vivía cómo un salvaje en la soledad.}
\pregunta{Enunciado con ortografía CORRECTA.}{null}
\opciones{¿Oyes? allá afuera está lloviendo. ¿No sientes el golpear de la lluvia? J. Rulfo}{¿Oyes? allá afuera está lloviendo. ¿No sientes el golpear de la lluvia? [J. Rulfo]}{Oyes? Allá afuera está lloviendo. ¿No sientes el golpear de la lluvia? (J. Rulfo)}{¿Oyes? Allá afuera está lloviendo. ¿No sientes el golpear de la lluvia? (J. Rulfo)}
\pregunta{Identifica los sinónimos. Relaciona\\
1.Ruborizar.\\
2. Cuesta.\\
3. Presagio.\\
4. Casero.\\
5. Comentar.\\
a. Hablar\\
b. Pendiente\\
c. Desintegrar.\\
d. Arrendador.\\
e. Vatinicio.\\
f. Avergonzar.}{null}
\opciones{1c, 2b, 3e, 4d, 5f}{1a, 2b, 3e, 4d, 5a}{1f, 2b, 3e, 4d, 5a}{1c, 2e, 3d, 4d, 5f}{1f, 2b, 3d, 4e, 5f}
\pregunta{Selecciona el inciso que contenga las palabras escritas correctamente.}{null}
\opciones{Huida, tráquea, búho, Beatriz, léemelo.}{Huída, traquea, búho, Beatríz, leemelo.}{Huida, tráquea, buho, Beatriz, lémelo.}{Huída, tráquea, buho, Beatríz, lemelo.}{Huida, tráquea, búho, Beatriz, léemelo.}
\pregunta{Relaciona según la clase de palabra que corresponda. Se han suprimido las tildes de la primera parte.\\
1.Dieciseis\\
2.Restamelo\\
3.Marmol\\
4.Espiritú\\
5.Arbol\\
a. Esdrújula\\
b. Grave\\
c. Aguda\\
d. Sobreesdrújula\\}{null}
\opciones{1a, 2b, 3d, 4c, 5b}{1b, 2c, 3a, 4d, 5c}{1c, 2a, 3b, 4a, 5c}{1d, 2c, 3b, 4b, 5a}{1c, 2d, 3b, 4a, 5b}
\pregunta{Selecciona la oración verdadera.}{null}
\opciones{Las palabras esdrújulas llevan tilde, excepto cuando terminan en vocal, n o s.}{Las palabras agudas llevan tilde cuando terminan en consonante que no sea n o s.}{El español siempre emplea tilde para señalar la sílaba tónica.}{En una palabra el acento prosódico recae sobre la sílaba tónica.}{Cuando las palabras graves terminan en s precedidas de consonante no llevan tilde}
\pregunta{Selecciona el inciso que contenga únicamente palabras agudas.}{null}
\opciones{Héctor, está, jamás, portal, título.}{Reloj, París, telón, consomé, catedral.}{Huerto, además, planta, botella, anillo.}{Examen, portón, estación, libro, montón.}{Banco, librero, silicón, también, botón.}
\pregunta{Elige la función de la lengua que predomina en el siguiente ejemplo. Luisa, ¿puedes limpiar la mesa y lavar los trastes por favor?}{null}
\opciones{Metalingüítica.}{Apelativa.}{Referencial}{Sintomática}
\end{multicols*}
\end{document}