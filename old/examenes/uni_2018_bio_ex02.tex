\documentclass[10pt,a4paper]{article}
	\usepackage{../../miconfiguracion}
		\configPage

\begin{document}
	\BgThispage
	\makemytitle{BIOLOGIA}{Segundo examen}
	\datosalu
		
	%SECCION DE PREGUNTAS
	\begin{multicols*}{2}
	
	\pregunta{Los experimentos realizados por
Louis Pasteur en el siglo XIX refutaron
rotundamente la tesis de la}
			{null}
		\opciones{evolución filética}
				{generación espontánea.}
				{reproducción diferencial}
				{inmunidad regenerada.}
				
				\pregunta{La Teoría Sintética explica que la
evolución es el fruto de la interacción
de la selección natural y la}
			{null}
		\opciones{clonación.}
				{propagación.}
				{variación genética.}
				{especiación.}
				
				\pregunta{Característica que define a los
productores primarios de un
ecosistema.}
			{null}
		\opciones{Degradación del material orgánico
hasta moléculas simples.}
				{Consumo de materia orgánica para la
obtención de energía.}
				{Captación de energía y producción de
materia orgánica}
				{Consumo de animales para la
obtención de nutrientes.}
				
				\pregunta{Fase del ciclo hidrológico que
reabastece de agua a los ecosistemas
continentales.}
			{null}
		\opciones{Escurrimiento hacia cuencas
oceánicas.}
				{Evaporación del agua superficial de
los mares.}
				{Filtración de agua hacia el subsuelo
marino.}
				{Condensación y precipitación sobre
montañas.}

\pregunta{La siguiente teoria fue postulada por 
 "Las especies permanecen inmutables de sus origenes, no evolucionan"}
			{null}
		\opciones{James Hutton}
				{Carlos Darwin}
				{Jean Lamarck}
				{Georges Cuvier}
				
				
				\pregunta{La siguiente teoria fue postulada por 
"Las leyes que rigen a la naturaleza no cambian"}
			{null}
		\opciones{James Hutton}
				{Carlos Darwin}
				{Jean Lamarck}
				{Georges Cuvier}
				
				\pregunta{Una mutación conlleva cambios en el
\line(1,0){40} de un individuo y genera
cambios a simple vista en su
\line(1,0){40}.}
			{null}
		\opciones{genotipo – cariotipo}
				{fenotipo – genoma}
				{genotipo – fenotipo}
				{fenotipo – cariotipo}

\pregunta{Los protozoarios se clasifican con
base en}
			{null}
		\opciones{los organelos de locomoción}
				{el tipo de respiración.}
				{su forma de nutrición.}
				{su forma de reproducción.}
				
				\pregunta{Unidad basica de la herencia}
			{null}
		\opciones{gen}
				{alelo}
				{fenotipo}
				{cromosoma}
				
				\pregunta{Organelo celular formado por quitina ,presente en hongos}
			{null}
		\opciones{ribosomas}
				{cloroplastos}
				{pared celular}
				{lisisimas}
				
				\pregunta{Es una extensa red de canales o cisternas ramificadas , limitada por membranas:}
			{null}
		\opciones{cloroplastos}
				{lisosoma}
				{ribosoma}
				{reticulo endoplasmatico}
				
				
				\pregunta{La función del ARNt es}
			{null}
		\opciones{transportar un aminoácido específico
al polipéptido en crecimiento.}
				{inhibir la expresión de un gen.}
				{llevar la información de la secuencia
de aminoácidos a la proteína.}
				{realizar las reacciones de catálisis en
los ribosomas.}
				
				
				\pregunta{Forman el primer eslabon en una cadena alimenticia}
			{null}
		\opciones{productores}
				{consumidores}
				{descomponedores}
				{depredadores}
				
				\pregunta{Quien postulo la ley de segregacion independiente}
			{null}
		\opciones{darwin}
				{marx}
				{mendel}
				{pasteur}
				
				\pregunta{Las formas de reproduccion asexual son}
			{null}
		\opciones{por gemacion, por fragmentacion de filamentos , por estolones o rizomas}
				{una serie de divisiones que dan origen a las esporas}
				{division de un organismo en 2 celulas hijas del mismo tamaño}
				{un organismo se reproduce al separarse en dos o mas fragmentos}
				
				\pregunta{Es una ventaja de la reproducción
sexual.}
			{null}
		\opciones{Permitir la estabilidad genética.}
				{Producir únicamente células
diploides.}
				{Ser fuente de variación genética.}
				{Permitir la estabilidad fenotípica.}
				
				\pregunta{En el proceso fotodependiente de la
fotosíntesis, la clorofila participa
directamente en la}
			{null}
		\opciones{combinación de CO2 y H2O.}
				{producción de glucosa.}
				{captación de energía lumínica.}
				{producción de oxígeno}
				
				
				\pregunta{Los citocromos de la cadena
respiratoria se caracterizan por la
capacidad de}
			{null}
		\opciones{sintetizar ATP.}
				{transportar electrones}
				{producir CO2.}
				{reducir el oxígeno}
				
				\pregunta{Actividad humana que deteriora la composicion quimica del suelo}
			{null}
		\opciones{crianza comercial de ganado vacuno y porcino}
				{desecho de productos industriales en rellenos sanitarios}
				{consumo de combustibles fosiles}
				{Tala de especies en peligro de extinsion}
				
				
				
				
				
				
				
				
				



				
				
					
	
	\end{multicols*}
	
\end{document}