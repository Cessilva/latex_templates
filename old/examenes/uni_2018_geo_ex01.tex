\documentclass[10pt,a4paper]{article}
	\usepackage{../miconfiguracion}
		\configPage
\begin{document}
	\BgThispage
	\makemytitle{Geografía}{Primer examen}
	\datosalu
	%\marcaagua
		
	%SECCION DE PREGUNTAS
	\begin{multicols*}{2}
\pregunta{Cuando se hace referencia a la distribución espacial de los hechos y fenómenos geográficos, se aplica el principio de}{null}
\opciones{localización.}{relación.}{extensión.}{generalización.}{conexión.}
\pregunta{Relaciona los siguientes países con los hemisferios que les corresponden.\\
Paises\\
I. Japón.\\
II. Angola.\\
III. Bolivia.\\
IV. Australia.\\
Hemisferios\\
a. Boreal.\\
b. Austral.\\
c. Oriental.\\
d. Occidental.}{null}
\opciones{I: b,d. II:a, d. III: a, c.IV: a, d.}{I: a, c. II: b, c. III b, d.IV: b, c.}{I: a, d. II: b ,c. III: b, c.IV: a, c}{I: b, c. II: a, d. III: a, d.IV: b, d}{I: a, c. II: b, d. III: a, d.IV: b, d}
\pregunta{Las entidades que atraviesan la Sierra Madre del sur}{null}
\opciones{Chiapas y Veracruz.}{Veracruz y Tlaxcala.}{Tlaxcala y Oaxaca.}{Oaxaca y Guerrero.}{Guerrero y Chiapas}
\pregunta{El agua subterránea se emplea principalmente para}{null}
\opciones{riesgo en zonas húmedas.}{construcción de presas.}{control de inundaciones.}{riesgo en áreas secas.}{producción de energía.}
\pregunta{En losúltimos 20 años, la región natural que ha sufrido la mayor alteración por el hombre es}{null}
\opciones{la región mediterránea.}{el desierto africano.}{la pardera australiana.}{el bosque canadiense.}{la selva brasileña.}
\pregunta{El cambio climático global generado por el efecto invernadero puede ocasionar:\\
I. Aumento de temperatura.\\
II. Disminución de temperatura.\\
III. Fusión de hielos.\\
IV. Disminución del nivel marino.\\
V. Severas inundaciones.}{null}
\opciones {I, II y III}{I, II y IV}{II, III, IV}{III, IV y V}{I, III y V}
\pregunta{En la distribución poblacional actual de México influye}{null}
\opciones{la existencia de amplios litorales.}{la actividad ganadera del noroeste.}{el desarrollo de la frontera norte.}{la pesca en las costas del Golfo}{el desarrollo petrolero del sureste.}
\pregunta{Son algunas características que distinguen a un país subdesarrollado:\\
I. Desarrollo económico dependiente.\\
II. Desarrollo ecónomico independiente.\\
III. Importanción de materias primas agropecuarias.\\
IV.Exportación de productos tropicales.\\
V. Importanción de productos manufacturados.}{null}
\opciones{I, II y III}{II, III y IV}{III, IV y V}{I, IV y V}{II, III y V}
\pregunta{La región de los Balcanes tuvo grandes cambios territoriales debido a}{null}
\opciones{la separación de Chechenia.}{la unificación alemana.}{el expansionismo de Rumania.}{el conflicto de Turquía y Bulgaría.}{la desintegración de Yugoslavia.}
\pregunta{La principal zona de producción de pétroleo en México es la}{null}
\opciones{costa del noreste.}{costa del Pacífico norte.}{región del noreste.}{costa del Golfo.}{región del Pacífico sur.}
\pregunta{La principal finalidad de Geografía moderna es}{null}
\opciones{describir los elementos naturales y sociales del medio geográfico.}{diferenciar las causas y efectos de los hechos y fenómenos geográficos.}{conocer los efectos terrestres povocados por los fenómenos naturales.}{explicar la relación entre los elemntos naturales y sociales del medio geográfico.}{localizar los elementos naturales y sociales sobre la superficie terrestre.}
\pregunta{Si vas de la Ciudad de México a Cuernavaca, la cadena montañosa que tienes que atravesar es la Sierra}{null}
\opciones{Madre del sur.}{Madre occidental.}{de la breña}{Madre Oriental.}{Volcánica Transversal.}

\pregunta{Las aguas continentales representadas por los ríos, lagos y aguas subterráneas, tienen su origen en}{null}
\opciones{el clima de los lugares.}{las corrientes marinas.}{el ciclo hidrológico.}{las mareas vivas y muertas.}{el tectoismo activo.}
\pregunta{La taiga es una región natural que se localiza en el}{null}
\opciones{sur de Estados Unidos de América, cenro de Europa y Malasia.}{sur de Filipinas, centro de México y Egipto.}{norte de chile, norte de Suráfrica y Australia.}{norte de México, sur de Italia y sur de India.}{centro de Canadá, norte de Europa y Siberia.}
\pregunta{El aumento y retención del bióxido de carbono en la parte inferior de la atmósfera, ocasionado por la actividad industrial y los medios de transporte, están generando}{null}
\opciones{el cambio global del agua.}{la alteración de los ríos}{el cambio climático global.}{la alteración del viento.}{la modificación del relieve.}
\pregunta{Estados de la República Mexicana con baja densidad de población:}{null}
\opciones{Quintana Roo y Veracruz.}{Veracruz y Oaxaca.}{Oaxaca y Morelos.}{Morelos y Baja California Sur.}{Baja California Sur y Quintana Roo.}
\pregunta{La Unión Europea tiene como propósitos fundamentales la}{null}
\opciones{libre circulación de mercancías, personas y políticas comunes.}{unificación de la modena y la creación de un fideicomiso agrario.}{utilización sin restricciones del espacio aéreo y del mar patrimonial.}{la protección de los derechos de los trabajadores y un salario uniforme.}{explotación sin restricciones de los recursos del carbón y del petróleo.}
\pregunta{En México, el litoral de mayor longitud corresponde al}{null}
\opciones{Pacífico.}{Atlántico.}{Golfo de México.}{Mar Caribe.}{Mar de las Antillas.}
\pregunta{Zona marítima donde se obtiene petróleo, aceite y gas:}{null}
\opciones{Poza Rica.}{Ciudad Madero.}{Ciudad Pemex.}{Salina Cruz.}{Sonda de Campeche.}
\pregunta{En el mundo las mayores reservas de energéticos se localizan en}{null}
\opciones{Norteamérica}{Lejano Oriente.}{Sureste de Asia.}{Medio Oriente.}

\end{multicols*}

\end{document}
