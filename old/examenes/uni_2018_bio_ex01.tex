\documentclass[10pt,a4paper]{article}
	\usepackage{../miconfiguracion}
		\configPage

\begin{document}
	\BgThispage
	\makemytitle{Biología}{Primer examen}
	\datosalu
	%\marcaagua
		
	%SECCION DE PREGUNTAS
	\begin{multicols*}{2}
	
	\pregunta{Las celulas son unidades . . .}{null}
		\opciones{centrales y anatomicas}
				{locales y energeticas}
				{originales y microscopicas}
				{funcionales y logicas}
				{estructurales y fisiologicas}
				
	
		\pregunta{Es la sustancia que interviene en las transacciones energeticas de la celula}{null}
	\opciones{AMP}
				{NAD}
				{FAD}
				{ATP}
				{GTP}
				
				\pregunta{¿Cual es la teoria de Lamark sobre la evolucion de las especies?}{null}
	\opciones{La seleccion natural y la mutacion son las causas de la evolucion de las especies}
				{Los organismos actuales son el resultado de un largo proceso de adaptacion y seleccion}
				{Los organismos mejor adapatados son los que dejan mayor numero de descendientes}
				{Las mutaciones beneficas son seleccionadas pues favorecen la adaptacion de las especies}
				{La herencia de los caracteres adquiridos y la ley del uso y el desuso de los organos}
				
				\pregunta{La evolucion prebiotica fue controlada por }{null}
	\opciones{La temperatura y el tiempo}
				{El tiempo y la atmosfera primitiva}
				{La presion y el clima}
				{El calor y la presion}
				{La atmosfera primitiva y el clima}
				
				\pregunta{Estructura de forma alargada constituida por una doble membrana que interviene en la sintesis del ATP}{null}
	\opciones{Lisosoma}
				{Aparato de golgi}
				{Ribosoma}
				{Mitocondria}
				{GTP}

\pregunta{En la actualidad es la teoria mas aceptada sobre el origen de los seres vivos}{null}
	\opciones{Creacionista}
				{Endosimbiotica}
				{Vitalista}
				{Quimiosintetica}
				{Protestista}
	
	\pregunta{Fases de la mitosis}{null}
	\opciones{Telofase,profase,anafase y metafase}
				{Interfase,anafase,telofase y metafase}
				{Metafase,profase, telofase y anafase}
				{Anafase,metafase,profase y telofase}
				{Profase,metafase,anafase y telofase}
				
				\pregunta{Nombre del proceso biologico que surgio hace aproximadamente 2mil millones de años y que aumento la concentracion de oxigeno libre en la atmosfera }{null}
	\opciones{Combustion}
				{Fotolisis}
				{Respiracion}
				{Fotosintesis}
				{Interperizacion}
				
				\pregunta{Inician el flujo de energia a traves de una cadena alimenticia en un ecosistema }{null}
	\opciones{consumidores}
				{desintegradores}
				{productores}
				{transformadores}
				{decomponedores}
				
				\pregunta{Es una ventaja de la reproduccion sexual}{null}
	\opciones{Permitir la estabilidad gentetica}
				{Producir unicamente celular diploides}
				{Ser fuente de variacion genetica}
				{Permitir la estabilidad fenotipica}
				{Producir estructuras celulares}
				
				\pregunta{Los citocromos de la cadena respiratoria se caracteriza por la capacidad de}{null}
	\opciones{sintetizar ATP}
				{transportar electrones}
				{producir dioxido de carbono}
				{reducir el oxigeno}
				{producir GTP}
				
				\pregunta{Los nucleotidos son unidades formadas por }{null}
	\opciones{un acido nucleuico, un azucar pentosa y un grupo fosfato}
				{un acido desoxirribonucleico,un azucar pentosa y un fosfato}
				{una base nitrogenada, un azucar hexosa y un grupo fosfato}
				{una base nitrogenada , un azucar pentosa y un grupo fosfato}
				{un acido nitrico ,un azucar pentosa y un fosfato}
				
				\pregunta{Tipos de gametos producidos por un progenitor con genotipo Aa - Bb}{null}
	\opciones{AA-BB-aa-bb}
				{AB-Ab-aB-ab}
				{AA-Aa-BB-Bb}
				{Aa-Bb-aa-bb}
				{Ab-Bb-AA-bb}
				
				\pregunta{Conjunto de organismos que comparten  la misma area fisica y se reproducen entre si forman}{null}
	\opciones{una poblacion}
				{un ecosistema}
				{un bioma}
				{una raza}
				{una celula}
				
				\pregunta{Una celula presenta un numero cromososmico con 46xx.Al dividirse durante la mitosis ,formara celulas con}{null}
	\opciones{23X}
				{23XX}
				{46XX}
				{46XY}
				{23Y}
				
				\pregunta{Ejemplos de enfermedades causadas por mutaciones en genes que codifican proteinas:}{null}
	\opciones{hemofilia y fenilcetonuria}
				{cancer y diabetes}
				{fenilcetonuria y SIDA}
				{hemofilia y cancer}
				{raquitismo y diabetes}
				
				\pregunta{los organismos pertenecientes a los reinos fungi,vegetal y animal presentan celulas de tipo}{null}
	\opciones{procariotas}
				{heteretrofos}
				{eucariotas}
				{autotrofas}
				{vegetales}
				
				\pregunta{El proceso fotosintetico que realizan los organismos es importante para los seres vivoss porque produce moleculas para la vida como }{null}
	\opciones{glucosa y bioxido de carbono}
				{oxigeno y bioxido de carbono}
				{glucosa y oxigeno}
				{oxigeno y fosforo}
				{oxigeno y mercurio}
				
				
				\pregunta{las mutaciones son importantes porque }{null}
	\opciones{eliminan organismos debiles y enfermizos}
				{producen la variabilidad sobre la que opera la seleccion natural}
				{originan variabilidad no heredable}
				{se producen en los individuos menos aptos}
				{se produce en los gametos}
				
				
				\pregunta{numero de cromosomas que tienen los gametos masculinos: }{null}
	\opciones{22}
				{23}
				{44}
				{45}
				{46}
				
				
				
				
				
	\end{multicols*}
	
\end{document}