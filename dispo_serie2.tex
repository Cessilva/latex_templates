\documentclass{mylib/reporteCorto}
\usepackage{float}
\usepackage{enumitem}
\usepackage{amsmath}

\title{Reporte}
\author{rodrigofranciscopablo }

\subject{Dispositivos y Circuitos Electrónicos}
\mytitle{Serie 2}
\mysubTitle{Serie del diodo semiconductor y modelos}
\students{Francisco Pablo \textsc{Rodrigo}}
\teacher{Dra. Moumtadi \textsc{Fatima}}
\group{10}
\deliverDate{13 de marzo de 2019}

\newcommand{\insertImage}[2]{
	\begin{figure}[H]
		\centering
		\includegraphics[width=#2cm]{#1}
	\end{figure}
}

\begin{document}

\coverPage

\tableofcontents
\newpage

\section{Ejercicios aplicaciones del diodo}

\subsection{Ejercicio 1}

Calcular la corriente, la tensión y la potencia en la carga, así como la potencia del diodo y la potencia total para
el circuito de la figura. Hacerlo considerando al diodo como :

\begin{enumerate}[label=\Alph*]
	\item Diodo ideal
	\item Diodo de sicilio
\end{enumerate}

\insertImage{img/dispos_serie2/appdiodo1}{15}

% Solucion 1

a) El diodo es ideal por lo que podemos tomar el diodo como un circuito corto, entonces

\begin{gather*}
	15 V = 470 \Omega * I \rightarrow I = 32 mA \\
	P_D = 0 W \\
	V_c = 470 \Omega *32 mA = 15 V \\
	P_c = 15V * 32 mA = 480 mW \\
	P_T = 15V * 32 mA = 480 mW
\end{gather*}

b) El diodo aquí es de 0.7 V

\begin{gather*}
	15 V = 470 \Omega * I + 0.7 V \rightarrow I = 32 mA \\
	P_D = 0.7V * 32.42 mA = 21.3 mA \\
	V_c = 470 \Omega *30.42 mA = 14.3 V \\
	P_c = 14.3V * 30.42 mA = 435 mW \\
	P_T = 15V * 30.42 mA = 456.3 mW
\end{gather*}

\subsection{Ejercicio 2}

Para el siguiente circuito:
\begin{enumerate}[label=\Alph*]
	\item Dibuja la señal de salida de $V_o$
	\item Calcular IL
\end{enumerate}

\insertImage{img/dispos_serie2/appdiodo2}{15}

Para el inciso a)

\insertImage{img/dispos_serie2/appdiodo2_a}{8}

Para el inciso b)

\begin{gather*}
	V_p = (\sqrt{2})(RMS) = 28.28 \\
	V_o = - \dfrac{V_p}{\pi} = \dfrac{-28.28}{\pi} = -9 V \\
	I_l = \dfrac{V_{CD}}{R} = \dfrac{-9 V}{2 \Omega} = -4.5 mA
\end{gather*}

\subsection{Ejercicio 3}

Para el siguiente circuito determinar
la onda de salida v0 que circula por la
resistencia central , tome en cuenta
que el circuito es un rectificador de
onda completa
El voltaje de entrada es de 14.4 Vp ,
los diodos son de silicio , el
voltímetro o v0 está conectado en
serie a la resitencia del centro del
circuito.
Respuesta: El voltaje de salida es 6.85
Vp tanto en el semiciclo positivo con
el el negativo

\insertImage{img/dispos_serie2/appdiodo3}{10}

El voltaje de entrada es 14.4 V y se le debe restar 0.7 V por el diodo y lo que tenemos es
13.7 V y mediante divisor de voltaje se puede resolver utilizando la fórmula

$$ V_x = V_T \left(\dfrac{R_x}{R_T}\right)$$

$V_x =$ Punto propuesto para divisor de voltaje\\
$V_T =$ Voltaje total (13.7 V)\\
$R_x =$ Resistencia para el cual se le aplicó el divisor (2 K $\Omega$ )\\
$R_T =$ Valor de la resistencia del análisis (4K $\Omega$ )\\

$V_o = (13.7 V)(2k\Omega / 4k \Omega) = 6.85$\\

Como es un circuito rectificador de onda completa se sabe que el semiciclo negativo es
igual al positivo con 6.85 V

$V_o = (-13.7 V)(2k\Omega / 4k \Omega) = -6.85$\\

\subsection{Ejercicio 4}

Calcular la corriente que circula
a través del diodo zener para la
siguiente
configuración
del
circuito así como su potencia y
concluir respecto a esta, si el
diodo zener no está activo
cambiar la configuración para
que este quede activo.

\insertImage{img/dispos_serie2/appdiodo4}{10}

Es necesario utilizar la medición de Thevenin para ver si está activo el diodo.\\

$V_{TH} = V_i (R_L / R_S R_L) = 22(4.7 k\Omega / 5.7 k\Omega) = 18.14 V$

Con leyes de Kirchhof

\insertImage{img/dispos_serie2/sol_appdiodo4}{10}



\section{Ejercicios de modelo a pequeña señal}

\subsection{Ejercicio 1}

\insertImage{img/dispos_serie2/seg_peque1}{15}

\insertImage{img/dispos_serie2/sol_seg_peque1}{10}

\subsection{Ejercicio 2}

\insertImage{img/dispos_serie2/seg_peque2}{15}

\insertImage{img/dispos_serie2/sol_seg_peque2}{10}

\subsection{Ejercicio 3}

\insertImage{img/dispos_serie2/seg_peque3}{15}

\insertImage{img/dispos_serie2/sol_seg_peque3}{10}



\section{Ejercicios de gran señal}

\subsection{Ejercicio 1}

\insertImage{img/dispos_serie2/seg_grande1}{15}

\insertImage{img/dispos_serie2/sol_seg_grande1}{10}

\subsection{Ejercicio 2}

\insertImage{img/dispos_serie2/seg_grande2}{15}

\insertImage{img/dispos_serie2/sol_seg_grande2}{10}


\subsection{Ejercicio 3}

\insertImage{img/dispos_serie2/seg_grande3}{15}

\insertImage{img/dispos_serie2/sol_seg_grande3}{10}


\subsection{Ejercicio 4}

\insertImage{img/dispos_serie2/seg_grande4}{15}

\insertImage{img/dispos_serie2/sol_seg_grande4}{10}


\subsection{Ejercicio 5}

\insertImage{img/dispos_serie2/seg_grande5}{15}

\insertImage{img/dispos_serie2/sol_seg_grande5}{10}



\section{Ejercicios de especificaciones del fabricante}

\subsection{Ejercicio 1}

Determine el voltaje pico inverso repetitivo para cada uno de los diodos 1N4002, 1N4003, 1N4004, 1N4005,  1N4006. 

\subsection{Ejercicio 2}

Si la corriente de polarización en directa es de 800 mA y el votaje de polarización en directa es de 0.75V en un 1N005, ¿excede la potencia nominal=

\subsection{Ejercicio 3}

¿Cuál es $I_{F(AV)}$ para un 1N4001  a temperatura ambiente de $100 ^\circ C$?

\subsection{Ejercicio 4}

¿Cuál es $I_{FSM}$ para un 1N4003 si el cambio súbito de corriente se repite 40 veces a 60 HZ?

\section{Ejercicios con uso de computadora}

\insertImage{img/dispos_serie2/sol_eje_fabricante}{10}


\subsection{Ejercicio 1}

Comprobar resultados obtenidos en el problema 1 de diodos el inciso b), incluir captura de pantalla verificando
y comparando los valores obtenidos.

\insertImage{img/dispos_serie2/dol_comp1}{10}


\subsection{Ejercicio 2}

Comprobar resultados obtenidos en el problema 4 (diodo zener), incluir captura de pantalla verificando y
comparando los valores obtenidos.

\insertImage{img/dispos_serie2/dol_comp2}{10}


\end{document}
