\documentclass{mylib/reporteConCalif}
\title{Reporte}
\author{rodrigofranciscopablo }

\subject{Laboratorio de Dispositivos y circuitos electrónicos}
\mytitle{Reporte de práctica 4}
\mysubTitle{Circuitos rectificadores}
\students{Francisco Pablo \textsc{Rodrigo}}
\teacher{M.I. Guevara Rodríguez \textsc{Ma. del Socorro}}
\group{8}
\deliverDate{20 de marzo de 2019}

\usepackage{float}
\usepackage{tabu}
\newcommand{\insertImage}[3]{
	\begin{figure}[H]
		\centering
		\includegraphics[width=#3cm]{#1}
		\caption{#2}
	\end{figure}
}

\begin{document}

\coverPage

\tableofcontents
\newpage

\section{Objetivos}

\subsection{General}

Analizar y diseñar circuitos electrónicos que contienen diodos semiconductores.

\subsection{Particular}

Analizar, diseñar, simular e implementar circuitos rectificadores de media onda y onda completa utilizando diodos de
propósito general.

\section{Introducción}

Un rectificador es el dispositivo electrónico que permite convertir la corriente alterna en corriente continua. Esto se realiza utilizando diodos rectificadores, ya sean semiconductores de estado sólido, válvulas al vacío o válvulas gaseosas como las de vapor de mercurio (actualmente en desuso).

Dependiendo de las características de la alimentación en corriente alterna que emplean, se les clasifica en monofásicos, cuando están alimentados por una fase de la red eléctrica, o trifásicos cuando se alimentan por tres fases.

Atendiendo al tipo de rectificación, pueden ser de media onda, cuando solo se utiliza uno de los semiciclos de la corriente, o de onda completa, donde ambos semiciclos son aprovechados.

El tipo más básico de rectificador es el rectificador monofásico de media onda, constituido por un único diodo entre la fuente de alimentación alterna y la carga.

\subsection{Rectificador de media onda}

Este es el circuito más simple que puede convertir corriente alterna en corriente continua. Este rectificador lo podemos ver representado en la siguiente figura:

\insertImage{img/labdisp_pract4/hwr}{Rectificador de media onda}{8}
	
Durante el semiciclo positivo de la tensión del primario, el bobinado secundario tiene una media onda positiva de tensión entre sus extremos. Este aspecto supone que el diodo se encuentra en polarización directa. Sin embargo durante el semiciclo negativo de la tensión en el primario, el arrollamiento secundario presenta una onda sinusoidal negativa. Por tanto, el diodo se encuentra polarizado en inversa.

\subsection{Rectificador de onda completa}

Un rectificador de onda completa es un circuito empleado para convertir una señal de corriente alterna de entrada (Vi) en corriente de salida (Vo) pulsante. A diferencia del rectificador de media onda, en este caso, la parte negativa de la señal se convierte en positiva o bien la parte positiva de la señal se convertirá en negativa, según se necesite una señal positiva o negativa de corriente continua.

\insertImage{img/labdisp_pract4/fwr}{Rectificador de onda completa}{8}


\newpage
\section{Previo}

\textbf{Diseña un circuito rectificador de media onda y uno de onda completa}

Datos
\begin{itemize}
	\item $R_L = 1k\Omega$
\end{itemize}

\subsection{Circuito rectificador de media onda}

	\insertImage{img/labdisp_pract4/hwrc}{Rectificador de media onda}{12}
	\begin{center}
	\begin{tabu} to 0.8\textwidth { | X[c] | X[c] | X[c] | X[c] | }
		 \hline
		 C & $V_{CD}$ & $V_{out}Rizo$ & $F_{out}Rizo$ \\
		 \hline
		 $1 \mu F$ & 21.2 V &  & 60 Hz \\
		 \hline
		 $10 \mu F$ & 21.2 V &  & 60 Hz \\
		 \hline
		 $100 \mu F$ & 21.2 V &  & 60 Hz \\
		 \hline
		 $1000 \mu F$ & 21.2 V &  & 60 Hz \\
		 \hline
	\end{tabu}
	\end{center}

	\insertImage{img/labdisp_pract4/ej1_1uf}{Circuito con capacitor de $1\mu F$}{12}


\subsection{Circuito rectificador de onda completa}

	\insertImage{img/labdisp_pract4/fwrc}{Rectificador de onda completa}{12}
	\begin{center}
	\begin{tabu} to 0.8\textwidth { | X[c] | X[c] | X[c] | X[c] | }
		 \hline
		 C & $V_{CD}$ & $V_{out}Rizo$ & $F_{out}Rizo$ \\
		 \hline
		 $1 \mu F$ & 21.2 V &  & 120 Hz \\
		 \hline
		 $10 \mu F$ & 21.2 V &  & 120 Hz \\
		 \hline
		 $100 \mu F$ & 21.2 V &  & 120 Hz \\
		 \hline
		 $1000 \mu F$ & 21.2 V &  & 120 Hz \\
		 \hline
	\end{tabu}
	\end{center}

	\insertImage{img/labdisp_pract4/ej2_1uf}{Circuito con capacitor de $1\mu F$}{12}


\newpage
\section{Desarrollo}

\subsection{Rectificador de media onda}

Para desarrollar esta part de la práctica diseñamos el circuito presentado a continuación y posperioromente fuimos analizando las diferentes capacitancias en él.

\insertImage{img/labdisp_pract4/hwrc}{Circuito con capacitor de $n\mu F$}{12}

\textbf{Para el caso sin capacitor obtuvimos}

\insertImage{img/labdisp_pract4/recm}{Circuito sin capacitor}{10}

\textbf{Para el caso de $C = 1 \mu F$}

\insertImage{img/labdisp_pract4/recm_1mf}{Circuito con $C = 1 \mu F$}{10}

\textbf{Para el caso de $C = 10 \mu F$}

\insertImage{img/labdisp_pract4/recm_10mf}{Circuito con $C = 10 \mu F$}{10}

\textbf{Para el caso de $C = 1000 \mu F$}

\insertImage{img/labdisp_pract4/recm_1000mf}{Circuito con $C = 1000 \mu F$}{10}

Con todos los valores obtenidos pudimos hacer la siguiente tabla.

\begin{center}
\begin{tabu} to 0.8\textwidth { | X[c] | X[c] | X[c] | X[c] | }
	 \hline
	 C & $V_{CD}$ & $V_{out}Rizo$ & $F_{out}Rizo$ \\
	 \hline
	 $1 \mu F$ & 21.05 V &  & 60 Hz \\
	 \hline
	 $10 \mu F$ & 21.05 V&  & 60 Hz \\
	 \hline
	 $100 \mu F$ & 21.05 V &  & 60 Hz \\
	 \hline
	 $1000 \mu F$ & 21.05 V &  & 60 Hz \\
	 \hline
\end{tabu}
\end{center}

\subsection{Rectificador de onda completa}	


Para desarrollar esta part de la práctica diseñamos el circuito presentado a continuación y posperioromente fuimos analizando las diferentes capacitancias en él.

\insertImage{img/labdisp_pract4/fwrc}{Circuito con capacitor de $n\mu F$}{12}

\textbf{Para el caso sin capacitor obtuvimos}

\insertImage{img/labdisp_pract4/recc}{Circuito sin capacitor}{10}

\textbf{Para el caso de $C = 1 \mu F$}

\insertImage{img/labdisp_pract4/recc_1mf}{Circuito con $C = 1 \mu F$}{10}

\textbf{Para el caso de $C = 10 \mu F$}

\insertImage{img/labdisp_pract4/recc_10mf}{Circuito con $C = 10 \mu F$}{10}

\textbf{Para el caso de $C = 1000 \mu F$}

\insertImage{img/labdisp_pract4/recc_1000mf}{Circuito con $C = 1000 \mu F$}{10}

Con todos los valores obtenidos pudimos hacer la siguiente tabla.

\begin{center}
\begin{tabu} to 0.8\textwidth { | X[c] | X[c] | X[c] | X[c] | }
	 \hline
	 C & $V_{CD}$ & $V_{out}Rizo$ & $F_{out}Rizo$ \\
	 \hline
	 $1 \mu F$ & 21.05 V &  & 120 Hz \\
	 \hline
	 $10 \mu F$ &21.05 V &  & 120 Hz \\
	 \hline
	 $100 \mu F$ &21.05 V &  & 120 Hz \\
	 \hline
	 $1000 \mu F$ &21.05 V &  & 120 Hz \\
	 \hline
\end{tabu}
\end{center}


\section{Conclusiones}

Observamos que tanto en el rectificador de media onda como en el de onda completa la presencia del capacitor fue notable ya que la señal obtenida con él fue bastante diferente a que si no lo tuviera.

Recordemos que el capacitor es un elemento que guarda energía por lo cuál ayudo a que nuestro circuito no bajará completamente su energía al tener una señal senoidal. Observamos que con el capacitor la energía iba bajando cada vez menos a medida que le poniamos un capacitor de mayor tamaño.

Por otra parte, comprobamos que el recortador de media onda efectivamente hace su trabajo y nos deja con solo la mitad de la señal y que el recortador de onda completa invierte la señal negativa y solo nos deja con la señal positiva, pero, muy importante, aumentando su frencuencia al doble.



\end{document}