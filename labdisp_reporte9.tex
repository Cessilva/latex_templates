\documentclass{mylib/reporteConCalif}
\graphicspath{ {img/labdisp_pract9/} }

\title{Reporte}
\author{rodrigofranciscopablo }

\subject{Laboratorio de Dispositivos y circuitos electrónicos}
\mytitle{Reporte de práctica 10}
\mysubTitle{Transistor de efecto de campo (FET) Caracterización}
\students{Francisco Pablo \textsc{Rodrigo}}
\teacher{M.I. Guevara Rodríguez \textsc{Ma. del Socorro}}
\group{8}
\deliverDate{15 de mayo de 2019}
\usepackage{mathtools}
\usepackage{amsmath}
\usepackage{float}
\usepackage{tabu}
\usepackage{subfig}

\begin{document}

\coverPage

%\tableofcontents
%\newpage

\section{Objetivos}

\subsection{General}

Analizar, diseñar circuitos amplificadores de una etapa con transistores de efecto de campo (FET).

\subsection{Particular}

Analizar, simular y caracterizar un FET, para identificar cada una de sus regiones de operación.

\section{Introducción}

El JFET (transistor de efecto de campo de unión) es un tipo de FET que opera con una unión \textit{pn} polarizada en inversa para contral corriente en un canal. Según su estructura, los JFET caen dentro de dos categorías, de canal n o de canal p.

\insertImage{img1}{Representación de un FET}{10}

En la figura observamos que cada extremo del canal n tiene una terminal; el \textbf{drenaje} se encuentra en el extremo superior de la \textbf{fuente} en el inferior. Se difunden dos regiones tipo p en el material tipo n para formar un \textbf{canal} y ambos tipos de regiones p se conectan a la terminal de la \textbf{compuerta}.


\newpage
\section{Previo}

\subsection{Curva de corriente de drenaje contra voltaje compuerta-fuente}

La curva de transconductancia del JFET es una gráfica que relaciona el $V_{GS}$ con el $I_D$

\insertImage{img2}{Relación de VGS con ID}{12}

\subsection{Curva de corriente de drenaje contra voltaje drenaje-fuente}

A continuación se presentan las curvas características del drenaje de JFET

\insertImage{img3}{Familia de curvas de caracterìstica del denaje}{10}



  \newpage
\section{Desarrollo}

Debido a que no hemos visto el funcionamiento de los FET en nuestras clases de teoría tuvimos que empezar con lo básico del FET que fue hacer $V_{G} = 0$ \\, una vez que hicimos esto obtuvimos los siguientes valores para 5,10 y 15 Volts en el voltaje $V_{DD}$

\textbf{$V_{DD} = 5 V$}
\begin{itemize}
  \item $I_D = 1.2 mA$
  \item $I_S = 1.2 mA$
\end{itemize}

\textbf{$V_{DD} = 10 V$}
\begin{itemize}
  \item $I_D = 3.1 mA$
  \item $I_S = 3.4 mA$
\end{itemize}

\textbf{$V_{DD} = 15 V$}
\begin{itemize}
  \item $I_D = 9.2 mA$
  \item $I_S = 9.1 mA$
\end{itemize}

Despues de entender que si el voltaje del gate es cero se comprobó que la corriente en drain y en source son practicamente iguales, ahora bien, comprobaremos que a medida que se incrementa el voltaje en el gate la corriente $I_D$ tenderá a disminuir hasta el punto que sea extremadamente pequeña o inclusive, igual a cero.

\insertImage{tablallena}{Tabla experimental del FET}{12}
\newpage
Entonces, podemos usar esta tabla para obtener la curva de caracterización del FET
\vspace{8cm}

\section{Conclusiones}

El transistor de efecto de campo de juntura o unión es un tipo de dispositivo electrónico de tres terminales que puede ser usado como interruptor electrónicamente controlado, amplificador o resistencia controlada por voltaje.\\

De la misma manera que los BJT, los FET tienen tres regiones de operación y por lo tanto tenemos que tener cuidado de en cual estamos trabajando ya que la región nos dirá como se comportará en transistor y por ende que función realizará.

\insertImage{curva}{Curva caracterìstica del FET}{8}

Se puede ver que para un determinado valor de voltaje de la puerta, la corriente es casi constante en un amplio rango de voltajes Fuente-Drenador. Cuando la puerta se hace más negativa, drena los portadores mayoritarios de la gran área de drenaje alrededor de la puerta. Esto reduce el flujo de corriente para un determinado valor de voltaje Fuente-Drenador. Modulando el voltaje de la puerta, se modula el flujo de corriente a través del dispositivo.



\end{document}
