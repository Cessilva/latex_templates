\documentclass[11pt]{mylib/documentoProteco}
\usepackage[T1]{fontenc}
\usepackage{multicol}
\org{\fontfamily{qag}\selectfont{PROGRAMA DE TECNOLOGÍA EN CÓMPUTO}}
\mytitle{\fontfamily{qag}\selectfont{Temario Android Básico}}
\mysubTitle{\fontfamily{qag}\selectfont{Intersemestral 2019-2}}
\deliverDate{\fontfamily{qag}\selectfont{junio de 2019}}

\begin{document}
\vspace*{4mm}
\fontfamily{qag}\selectfont

\setlength{\footskip}{60pt}

\renewcommand{\labelenumiii}{\roman{enumiii}}
\begin{multicols*}{2}
\begin{enumerate}
  \item Kotlin
  \begin{enumerate}
    \item ¿Por qué Kotlin?
    \item Tipos de variables
    \item Estructuras de control
    \item Manejo de errores y excepciones
    \item Funciones
    \item Clases e interfaces
    \item Particularidades del lenguaje
    \begin{enumerate}
      \item Null safety
      \item Elvis Operator
    \end{enumerate}
  \end{enumerate}

  \item Graphical User Interface (GUI)
  \begin{enumerate}
    \item Contenedores
    \begin{enumerate}
      \item LinearLayout
      \item RelativeLayout
      \item ConstraintLayout
    \end{enumerate}
    \item Widgets(TextView, EditText, Button, etc.)
    \item Orientación del dispositivo
    \item Tamaños de pantalla
    \item Soporte de idiomas
    \item Estilos y temas
  \end{enumerate}

  \item Actividades e Intents
  \begin{enumerate}
    \item Ciclo de vida de una actividad
    \item Intents explícito e implícito
    \item Transferencia de datos
    \item Android Native Actions
  \end{enumerate}

  \item Vistas dinámicas
  \begin{enumerate}
    \item GridView
    \item RecyclerView
  \end{enumerate}

  \item Persistencia de datos
  \begin{enumerate}
    \item SQLite básico
    \item Firebase
  \end{enumerate}

\end{enumerate}

\columnbreak
\noindent
NOTA: Preferentemente traer Android Studio instalado

\end{multicols*}

\end{document}
