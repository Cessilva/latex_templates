% FortySecondsCV LaTeX template
% Copyright © 2019 René Wirnata <rene.wirnata@pandascience.net>
% Licensed under the 3-Clause BSD License. See LICENSE file for details.
%
% Attributions
% ------------
% * fortysecondscv is based on the twentysecondcv class by Carmine Spagnuolo
%   (cspagnuolo@unisa.it), released under the MIT license and available under
%   https://github.com/spagnuolocarmine/TwentySecondsCurriculumVitae-LaTex
% * further attributions are indicated immediately before corresponding code


%-------------------------------------------------------------------------------
%                             ADDITIONAL PACKAGES
%-------------------------------------------------------------------------------
\documentclass[
  a4paper,
%   showframes,
%   maincolor=cvgreen,
%   sectioncolor=red,
%   subsectioncolor=orange
%   sidebarwidth=0.4\paperwidth,
%   topbottommargin=0.03\paperheight,
%   leftrightmargin=20pt
]{fortysecondscv}

% improve word spacing and hyphenation
\usepackage{microtype}
\usepackage{ragged2e}

% take care of proper font encoding
\ifxetex
	\usepackage{fontspec}
	\defaultfontfeatures{Ligatures=TeX}
% \newfontfamily\headingfont[Path = fonts/]{segoeuib.ttf} % local font
\else
	\usepackage[utf8]{inputenc}
	\usepackage[T1]{fontenc}
% \usepackage[sfdefault]{noto} % use noto google font
\fi

% enable mathematical syntax for some symbols like \varnothing
\usepackage{amssymb}

% bubble diagram configuration
\usepackage{smartdiagram}
\smartdiagramset{
  % defaut font size is \large, so adjust to harmonize with sidebar layout
  bubble center node font = \footnotesize,
  bubble node font = \footnotesize,
  % default: 4cm/2.5cm; make minimum diameter relative to sidebar size
  bubble center node size = 0.4\sidebartextwidth,
  bubble node size = 0.25\sidebartextwidth,
  distance center/other bubbles = 1.5em,
  % set center bubble color
  bubble center node color = maincolor!70,
  % define the list of colors usable in the diagram
  set color list = {maincolor!10, maincolor!40,
  maincolor!20, maincolor!60, maincolor!35},
  % sets the opacity at which the bubbles are shown
  bubble fill opacity = 0.8,
}


%-------------------------------------------------------------------------------
%                            PERSONAL INFORMATION
%-------------------------------------------------------------------------------
% profile picture
\cvprofilepic{pics/profile.png}
% your name
\cvname{Rodrigo Francisco}
% job title/career
\cvjobtitle{Software Developer Jr}
% date of birth
\cvbirthday{Jan 2, 1998}
% short address/location, use \newline if more than 1 line is required
\cvaddress{Park Ave.~1, 555 555 B-Woods}
% phone number
\cvphone{55 7729 9882}
% personal website
\cvsite{https://pandascience.net}
% email address
\cvmail{rhodfra@gmail.com}
% pgp key
\cvkey{4096R/FF00FF00}{0xAABBCCDDFF00FF00}
% add additional information
% \newcommand{\additional}{some more?}


%-------------------------------------------------------------------------------
%                              SIDEBAR 1st PAGE
%-------------------------------------------------------------------------------
% overwrite default icons and order of personal information
% \renewcommand{\personaltable}{%
% 	\begin{personal}[0.8em]
% 		\circleicon{\faKey}      & \cvkey  \\
% 		\circleicon{\faAt}       & \cvmail \\
% 		\circleicon{\faGlobe}    & \cvsite \\
% 		\circleicon{\faPhone}    & \cvphone \\
% 		\circleicon{\faEnvelope} & \cvaddress \\
% 		\circleicon{\faInfo}     & \cvbirthday \\
% 		% add another line
% 		\circleicon{\faQuestion} & \additional
% 	\end{personal}
% }

% add more profile sections to sidebar on first page
\addtofrontsidebar{
	% include gosquare national flags from https://github.com/gosquared/flags;
	% naming according to ISO 3166-1 alpha-2 country codes
	\graphicspath{{pics/flags/}}

	\profilesection{Languages}
		\pointskill{\flag{MX.png}}{Spanish}{5}
		%\pointskill{\flag{DE.png}}{German}{3}
  	\pointskill{\flag{GB.png}}{English}{4}
  	%\pointskill{\flag{FR.png}}{French}{3}

    \profilesection{Programming \\Languages}
  	\barskill{\faCode}{Kotlin,Python,C}{60}
  	\barskill{\faCode}{JavaScript, Java}{30}
  	\barskill{\faCode}{PHP}{20}

	\profilesection{Technologies \\ and Tools}
    \pointskill{\faMobile}{Android Studio}{3}
		\pointskill{\faGit}{Git}{3}
 			%\skill[1.8em]{\faCompress}{No need to specify further}
		\pointskill{\faTasks}{LaTeX}{4}
 			%\skill[1.8em]{\faCompress}{Having white parts}
 			%\skill[1.8em]{\faCompress}{Having black parts}
		\pointskill{\faHtml5}{HTML,CSS}{3}
 			%\skill[1.8em]{\faCompress}{On a tree}
 			%\skill[1.8em]{\faCompress}{On the grass}
    \pointskill{\faTerminal}{GNU/Linux}{4}
    \pointskill{\faDesktop}{Django}{3}
		\pointskill{\faMobile}{React Native}{1}
 			%\skill[1.8em]{\faCompress}{Kung Fu Panda 1-3}
 			%\skill[1.8em]{\faCompress}{WoW: Mists of Pandaria}
}


%-------------------------------------------------------------------------------
%                         TABLE ENTRIES RIGHT COLUMN
%-------------------------------------------------------------------------------
\begin{document}

\makefrontsidebar

\cvsection{Working Experience}
\begin{cvtable}
	\cvitem{currently}{Developer}{Ludomatics}{Focused on creating educational
  material on \LaTeX\, such as notes and exams. Also developed HOREDI system in
  order to make faster the exam qualification process. And created frontend
  website for the company (http://www.ludomatics.org)}
	\cvitem{2017 -- 2019}{Instructor and developer}{PROTECO}{
		PROTECO organization focus on giving courses and developing all sort of proyects
    related to computer technologies, so as I was part of this I gave several
    courses and are list down here
    \begin{itemize}
      \item GNU/Linux Basic course
      \item Java Basic course
      \item Django Basic course
      \item Android course with Kotlin
    \end{itemize}
    As part of this organization I also did coordinated various proyects like
    \begin{itemize}
      \item Remake of PROTECO website using Django
      \item External mobile application project from Lic. Jorge Cruz Gómez
    \end{itemize}
  }
\end{cvtable}


\cvsection{Education}
\subsection{Study}
\begin{cvtable}
	\cvitem{2016 -- }{Computer Engineering}{UNAM}
		{Currenly studying Computer Engineering at Engineering Faculty of National Autonomous University of México}
\end{cvtable}

\subsection{Courses}
\begin{cvtable}
	\cvitem{2017}{\textbf{C}. Deep understanding on how C language works}{PROTECO}{}
	\cvitem{2017}{\textbf{Web design}. Basic concepts of how to create a website}{PROTECO}{}
  \cvitem{2017}{\textbf{Java}. Introduction to java language and orientated object programming}{PROTECO}{}
  \cvitem{2017}{\textbf{Arduino}. Principles for building hardware proyects using manufactured boards }{PROTECO}{}
	\cvitem{2017}{\textbf{Networking}. Principles for understanding how our network works }{PROTECO}{}
\end{cvtable}


\cvsection{Projects}
\begin{cvtable}
	\cvitem{2018}{HOREDI}{integrated in http://www.ludomatics.org}{This systems provides and interface to check students exam answer sheet and give their grades as soon as they click on the submit button.}
	%\cvitem{2019}{ROOT-BRAIN}{location}{Some longer and more detailed
	%	description, that takes two lines	of space instead of only one.}
  \cvitem{2019}{Quine-McCluskey}{http://www.rhofp/Quine-McCluskey}{This project was developed with JavaScript and it basically emulates Quine-McCluskey boolean functions minimization method. }
\end{cvtable}

% \cvsection{Publications}
% \begin{cvtable}
% 	\cvitem{2010}{Cooking: 100 recipes for lazy Pandas}{Panda's
% 	Culinary World}{}
% 	\cvitem{2005}{Pandastasia}{Bamboo Books Assoc.}{}
% 	\cvitem{2000}{The Panda Way - A guide for mastering everyday life as a Panda}
% 		{Young Panda's Journal}{}
% \end{cvtable}


% \cvsection{Awards}
% \begin{cvtable}
% 	\cvitem{2010 -- now}{Panda of the Year}{Panda World Forum}{}
% 	\cvitem{2005 -- now}{Face of World Wide Fund for Nature}{WWF}{}
% 	\cvitem{2000}{Winner of Bamboo Sprouts Eating Contest}{Bamboo Society}{}
% \end{cvtable}


% \cvsection{Extra-Curricular Activities}
% \begin{cvtable}
% 	\cvitemshort{Relaxing}{Master the fine art of relaxing everywhere}
% 	\cvitemshort{Music}{Playing the bamboo flute in the 1st Panda Orchestra}
% 	\cvitemshort{Education}{Teaching young pandas to be more panda-like}
% \end{cvtable}

\cvsignature

\end{document}
