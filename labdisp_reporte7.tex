\documentclass{mylib/reporteConCalif}
\graphicspath{ {img/labdisp_pract7/} }

\title{Reporte}
\author{rodrigofranciscopablo }

\subject{Laboratorio de Dispositivos y circuitos electrónicos}
\mytitle{Reporte de práctica 7}
\mysubTitle{Transistor bipolar de juntura (TBJ) Caracterización}
\students{Francisco Pablo \textsc{Rodrigo}}
\teacher{M.I. Guevara Rodríguez \textsc{Ma. del Socorro}}
\group{8}
\deliverDate{10 de abril de 2019}
\usepackage{mathtools}
\usepackage{amsmath}
\usepackage{float}
\usepackage{tabu}
\usepackage{subfig}

\begin{document}

\coverPage

%\tableofcontents
%\newpage

\section{Objetivos}

\subsection{General}

Analizar y diseñar circuitos amplificadores de una etapa con transistores TBJ.

\subsection{Particular}

Caracterizar un TBJ, para identificar cada una de sus regiones de operación.

\section{Introducción}

El transistor de unión bipolar (del inglés bipolar junction transistor, o sus siglas BJT) es un dispositivo electrónico de estado sólido consistente en dos uniones PN muy cercanas entre sí, que permite aumentar la corriente y disminuir el voltaje, además de controlar el paso de la corriente a través de sus terminales. La denominación de bipolar se debe a que la conducción tiene lugar gracias al desplazamiento de portadores de dos polaridades (huecos positivos y electrones negativos), y son de gran utilidad en gran número de aplicaciones; pero tienen ciertos inconvenientes, entre ellos su impedancia de entrada bastante baja.

El BJT (transistor de unión bipolar) se construye con tres regiones semiconductoras se-
paradas por dos uniones pn.Un tipo se compone de dos regiones n separadas por
una región p (npn) y el otro tipo consta de dos regiones p separadas por una región n (pnp) como se puede ver en la imagen de abajo.

\insertImage{pre1}{Estructura de un TBJ}{15}

La imagen de abajo muestra los arreglos para polarización tanto de BJT npn como pnp para que ope-
ren como amplificador. Observe que en ambos casos la unión base-emisor (BE) está polarizada
en directa y la unión base-colector (BC) polarizada en inversa. Esta condición se llama polariza-
ción en directa-inversa.

\insertImage{pol}{Polarización directa-inversa del TBJ}{15}

\subsection{Construcción básica de un BJT.}

La unión pn que une la región de la base y la región del emisor se llama unión base-emisor.
La unión pn que une la región de la base y la región del colector se llama unión base-colector.
La región de la base está ligeramente dopada y es muy delgada en comparación con las regiones del emisor, excesivamente dopada, y la del colector, moderadamente dopada.

\insertImage{simb}{Símbolos de BJT estándar}{10}


\newpage
\section{Previo}

\insertImage{0_10}{Circuito con TBJ}{12}

\insertImage{tabla}{}{15}


\newpage
\section{Desarrollo}


\section{Conclusiones}



\end{document}
