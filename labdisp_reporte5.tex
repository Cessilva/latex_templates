\documentclass{mylib/reporteConCalif}
\title{Reporte}
\author{rodrigofranciscopablo }

\subject{Laboratorio de Dispositivos y circuitos electrónicos}
\mytitle{Reporte de práctica 5}
\mysubTitle{Circuitos recortador y multiplicador de tensión}
\students{Francisco Pablo \textsc{Rodrigo}}
\teacher{M.I. Guevara Rodríguez \textsc{Ma. del Socorro}}
\group{8}
\deliverDate{27 de marzo de 2019}

\usepackage{float}
\usepackage{tabu}
\newcommand{\insertImage}[3]{
	\begin{figure}[H]
		\centering
		\includegraphics[width=#3cm]{#1}
		\caption{#2}
	\end{figure}
}

\begin{document}

\coverPage

%\tableofcontents
%\newpage

\section{Objetivos}

\subsection{General}

Analizar y diseñar circuitos electrónicos que contienen diodos semiconductores.

\subsection{Particular}

Analizar, diseñar, simular e implementar circuitos recortador, sujetador y multiplicador de tensión utilizando diodos
de propósito general.

\section{Introducción}

\subsection{Circuito recortador}

Un limitador o recortador es un circuito que, mediante el uso de resistencias y diodos, permite eliminar tensiones que no nos interesen para que no lleguen a un determinado punto de un circuito. Mediante un limitador podemos conseguir que a un determinado circuito le lleguen únicamente tensiones positivas o solamente negativas.

Estos tipos de circuitos utilizan dispositivos de una o más uniones PN como elementos de conmutación. Se diseñan con el objetivo de recortar o eliminar una parte de la señal que se le introduce en sus terminales de entrada y permita que pase el resto de la forma de onda sin distorsión o con la menor distorsión posible. Para realizar esta función de recortar, los recortadores hacen uso de la variación brusca que experimenta la impedancia entre los terminales de los diodos y transistores al pasar de un estado a otro, de ahí que sean los elementos básicos en dichos circuitos. 

\insertImage{img/labdisp_pract5/recpos}{Circuito recortador}{8}
	
\subsection{Circuito sujetador}

Estos circuitos basan su funcionamiento en la acción del diodo, pero al contrario que los limitadores no modificarán la forma de onda de la entrada, es decir su voltaje o tipo de corriente eléctrica, sino que le añaden a ésta un determinado nivel de corriente continua. Esto puede ser necesario cuando las variaciones de corriente alterna deben producirse en torno a un nivel concreto de corriente continua.

\insertImage{img/labdisp_pract5/suj}{Rectificador de onda completa}{8}

\section{Circuito Multiplicador}

Un Multiplicador de tensión es un circuito eléctrico que convierte tensión desde una fuente de corriente alterna a otra de corriente continua de mayor voltaje mediante etapas de diodos y condensadores.

\insertImage{img/labdisp_pract5/mult}{Rectificador de onda completa}{8}


\newpage
\section{Previo}

\subsection{Diseña un circuito recortador positivo}

\subsection{Diseña un circuito recortador negativo}

\subsection{Diseña un circuito duplicador, triplicado y cuadriplicador de tensión}

\newpage
\section{Desarrollo}


\section{Conclusiones}


\end{document}