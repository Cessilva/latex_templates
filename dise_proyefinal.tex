\documentclass{mylib/reporte}
\graphicspath{ {img/dise_final/} }
\usepackage{float}
\title{Reporte}

\subject{Diseño Digital Moderno}
\mytitle{Proyecto Final}
\mysubTitle{Elevador de 4 pisos}
\students{
	Flores Martínez \textsc{Emanuel}\\
	Francisco Pablo \textsc{Rodrigo}\\
	García Ruíz \textsc{Andrea}
}
\teacher{Ing. Mandujano Wild \textsc{Roberto F.}}
\group{6}
\deliverDate{14 de Junio 2019}

\begin{document}

\coverPage

\newpage

\section{Introducción}

A lo largo del semestre 2019-2 aprendimos muchísimas cosas en la materia de diseño digital moderno, estudiamnos circuitos combinacionales y circuitos secuencias y junto con ellos vimos elementos de memoria como los \textit{flip-flop}, \textit{ROM}, etc.

\section{Objetivos}

Aplicar todo lo visto en la materia.

\section{Planteamiento del problema}

Para el proyecto final de la materia de diseño digital moderno nos dimos a la tarea de diseñar un elevador de 4 pisos "inteligente", el cual tiene 4 botones, uno por cada piso, entonces, si hay 2 personas en distintos pisos, por ejemplo 3 y 4, el usuario en el piso 3 aprieta su botón correspondiente y entonces el elevador 1 lo antenderá, luego si el usuario en el piso 4 presiona su botón y el elevador 1 no esta disponible corresponde al elevador 2 atender al usuario 2.

\section{Material}

\begin{enumerate}
  \item AND's
  \item OR'
  \item XOR's
  \item NOT's
  \item Contador de 4 bits
  \item Registro
  \item Comparador
  \item Puente H
  \item Codificador
  \item Fotoreceptores
  \item Fotoemisores
  \item Decodificador BCD
  \item 555
  \item Capacitores
  \item Push buttons
  
\end{enumerate}

\section{Patigrama}

\insertImage{dia1}{Diagrama del circuito parte 1}{17}
\insertImage{dia2}{Diagrama del circuito parte 2}{17}

\section{Implementación}

\insertImage{circuito}{Implementación del circuito}{15}

\insertImage{maqueta}{Maqueta hecha con AutoCAD}{15}

\section{Conclusiones}

Implementar el proyecto fue un verdadero reto pues la implementación fue demasiado complicada pues requerimos de muchos circuitos integrados, además, combinarlo con la parte mecánica resulóo particularmente interesante por nuestra casi nula experiencia con los mecanismos que hacen que el elevador funcione.

Aprendimos mucho sobre la importancia de las resistencias en nuestro dispositivos electrónicos, ya que estás hacen que el cero y el uno lógico deben de estar perfectamente asegurados.

\end{document}
