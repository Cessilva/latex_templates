\documentclass[10pt,a4paper]{article}
	\usepackage{miconfiguracion}
		\configPage

\begin{document}
	\BgThispage
	\makemytitle{Arítmetica}{Segundo examen}
	\datosalu
	%\marcaagua
		
	%SECCION DE PREGUNTAS
	\begin{multicols*}{2}
	
	\pregunta{A las 3 a.m. de un día de invierno se reporta una
			temperaturna de -3 °C, a las 12 del día la temperatura
			ya  es de 15 °C, ¿de cuántos grados fue la diferncia de
			temperaturas?}
			{null}
		\opciones{18 °C}
				{12 °C}
				{5 °C}
				{-12 °C}

	\pregunta{Para aprobar un examen de idiomas, un estudiante debe obtener
			promedio mínimo de 8 y calificaciones no menores a 7 en cada uno
			de los parciales. Juan tiene la siguientes calificaciones en los
			primeros 3 parciales: 7.2, 8.5 y 7.9. ¿Cuál es la mínima calificación
			que deberá obtener en el cuarto y último parcial para aprobar el curso?}
			{null}
		\opciones{6.9}
				{7}
				{8.4}
				{8.9}
	
	\pregunta{Una persona pesa 96 kg. Al ingresar a una clínica reductora de peso,
			comienza a bajar $\dfrac{1}{32}$ del peso inicial cada semana. ¿Cuántos
			kg pesará al término de cada una de las primeras cuatro semanas del 
			tratamiento?}
			{null}
		\opciones{Semana 1: 93\\
				Semana 2: 90\\
				Semana 3: 87\\
				Semana 4: 84}
				{Semana 1: 96\\
				Semana 2: 92\\
				Semana 3: 88\\
				Semana 4: 84}
				{Semana 1: 96\\
				Semana 2: 90\\
				Semana 3: 84\\
				Semana 4: 78}
				{Semana 1: 93\\
				Semana 2: 92\\
				Semana 3: 91\\
				Semana 4: 90}
	
	\pregunta{Al simplificar la expresión 
			$\left(\dfrac{3^4}{3^7}\right)^\frac{1}{3}$ se obtiene}
			{null}
		\opciones{$-3$}
			{$\dfrac{1}{3}$}
			{3}
			{$-\dfrac{1}{3}$}

	\pregunta{El estacionamiento supermercado tiene espacio para 1,000
			carros. El jueves hubo 200 autos compactos y algunos de tamaño 
			estándar. Elesacionamientos estuvo ocupado $\dfrac{3}{4}$ partes
			del total.\\
			¿Cuántos autos de tamaño estándar había en el estacionamiento el jueves?}
			{null}
		\opciones{500}
				{550}
				{600}
				{650}

	\pregunta{ $-5-\dfrac{3}{4}\left[-8+5\left(\dfrac{2}{3}-2\right)\right]$ 
			se obtiene}
			{null}
		\opciones{0}
				{$\dfrac{35}{3}$}
				{$6$}
				{$-\dfrac{35}{3}$}
				{-50}

	\pregunta{En una tienda de abarrotes, un empleado vende 3/5 de 
			una pieza de jamón de pierna y después 6/8 del resto. 
			¿Cuánto de jamón quedan. si la pieza entera pesa 6.0 Kg?}
			{null}
		\opciones{3.3 kg}
				{2.4 kg}
				{0.60 kg}
				{2.7 kg}
				{0.9 kg}

 	\pregunta{En la recta real, el número $\mathbf{\dfrac{7}{8}}$
			se encuentra entre los números}
			{null}
		\opciones{$\dfrac{11}{16}$ y $\dfrac{13}{16}$}
				{$\dfrac{15}{16}$ y $\dfrac{17}{16}$}
				{$\dfrac{25}{32}$ y $\dfrac{27}{32}$}
				{$\dfrac{27}{32}$ y $\dfrac{29}{32}$}
				{$\dfrac{53}{64}$ y $\dfrac{55}{64}$}

	\pregunta{Determina el resultado de la siguiente operación:
			$18+12\div6-3\times2$}
			{null}
		\opciones{30}
				{4}
				{-1}
				{14}
				{20}

	\pregunta{El resultado de la operación
		$(-4)^2-4\times(-2)\div(-2)^2+1$ es:}
			{null}
		\opciones{-3}
			{2}
			{-19}
			{18}
			{19}

	\pregunta{Determinar el m.c.m de 60,42 y 12 en términos de números primos}
			{null}
		\opciones{$2^2 \times 3 \times 5 \times 7$}
				{$2 \times 3 \times 5 \times 7$}
				{$2 \times 3^2 \times 5 \times 7$}
				{$2 \times 3 \times 5^2 \times 7$}
				{$2 \times 3 \times 5 \times 7^2$}
	
	\pregunta{Una señora tiene dos retazos de tela de 36 m y 48 m que quiere 
			dividir en pedazos iguales y de la mayor longitud posible. La 
			longitud de cada pedazo es:}
			{null}
		\opciones{144 m}
				{18 m}
				{24 m}
				{20 m}
				{12 m}

	\pregunta{La operación $(-3)^2-[|-7|-|6-8|-|-4|]$=}
			{null}
		\opciones{-8}
				{2}
				{13}
				{9}
				{8}

	\pregunta{Al simplificarse la operación 
		$\dfrac{1}{3}-\left[\dfrac{1}{2}+5\left(\dfrac{1}{5}+3\right)\right]$ se
		obtiene:}
			{null}
		\opciones{$\dfrac{97}{6}$}
				{$-\dfrac{97}{6}$}
				{$\dfrac{101}{6}$}
				{$-\dfrac{25}{6}$}
				{$-\dfrac{101}{6}$}

	\pregunta{En una empresa que renta maquinaria, fue alquilada una podadora a
			\$250.00 por hora. ¿Cuánto se debe cobrar si se alquila por 
			$18\dfrac{3}{5}$ de hora?}
			{null}
		\opciones{\$ 2,700.00}
				{\$ 4,650.00}
				{\$ 4,350.00}
				{\$ 900.00}
				{\$ 1,050.00}
	\pregunta{Si un vestido cuesta \$ 347.30 con IVA incluido
			entonces el precio del vestido sin el 15 \% del IVA es:}
			{null}
		\opciones{\$296}
				{\$300}
				{\$295.20}
				{\$302}
				{\$305.20}

	\pregunta{Un ejemplo de número irracional es:}
			{null}
		\opciones{$e = 2.718281...$}		
				{2.34343434...}
				{($\sqrt{2})^2$}
				{0.5}
				{$1.\overline{001}$}

	\pregunta{La suma de $\sqrt{-18}+\sqrt{-8}+2\sqrt{-50}$ es:}
			{null}
		\opciones{$-15\sqrt{2}i$}
				{$15\sqrt{2}i$}
				{$5\sqrt{2}i$}
				{$15i$}
				{$-5\sqrt{2}i$}

	\pregunta{Si $Z=2+5i$ entonces $Z^2$ es:}
			{null}
		\opciones{$-21+20i$}
				{$29+20i$}
				{$4+25i$}
				{$-21+10i$}
				{$29-20i$}
	\pregunta{Realiza la siguiente operación de números complejos.
			$(3-i)+6i-4(2+3i)$}
			{null}
		\opciones{$11-19i$}
				{$11-31i$}
				{$11-\sqrt{-19}$}
				{$-(8+33i)$}
				{$-(5+7i)$}
	
	
	\end{multicols*}
	
\end{document}