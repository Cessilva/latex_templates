\documentclass[10pt,a4paper]{article}
	\usepackage{../miconfiguracion}
  		\configPage

\begin{document}
	\BgThispage
	\makemytitle{Historia de México.}{Segundo examen}
	\datosalu
		
	%SECCION DE PREGUNTAS
	\begin{multicols*}{2}
\pregunta{La constitución que nos rige establece la educación laica en su artículo} {null}
		\opciones{1}
				{130}
				{27}
				{123}
				{3}
				\pregunta{Indica el orden cronológico en que se dieron los gobiernos de los siguientes presidentes 1:Miguel Alemán 2:Lázaro Cárdenas 3:Gustavo Díaz Ordaz 4:Luis Echeverría}{null}
		\opciones{4,3,2,1}
				{1,2,3,4}
				{3,1,2,4}
				{3,4,1,2}
				{2,1,3,4}
		\pregunta{Señala el acontecimiento que NO se haya llevado a cabo durante el gobierno cardenista}{null}
		\opciones{Expropiasión de la industria eléctrica}
				{Creación de IPN}
				{Expropiación de la industria petrolera}
			 {creación del Partido de la Revolución Mexicana}
			 {Creación de la confederación Nacional Campesina}
			 \pregunta{Entre las Leyes de Reforma promovidas por el gobierno de Benito Juárez, cabe mencionar la que estlablece el:}{null}
		\opciones{registro civil}
				{repudio a la intervención}
				{control de cambio}
				{control demográfico}
				{principio de no reelección}
				\pregunta{¿Cuál  fue el lapso abarcado por el periodo conocido como porfiriato}{null}
		\opciones{1876-1920}
				{1874-1902}
				{1857-1910}
				{1910-1917}
				{1876-1911}
		\pregunta{Ordena cronológicamente los acontecimientos que ocurrieron durante la intervención Francesa 1:Batalla del Cinco de Mayo. 2:Llegada de Maximiliano y Carlota a México. 3:Convención de Londres (Alianza Tripartita)4:Sitio de Queretaro.} {null}
		\opciones{4,3,1,2}
				{3,4,1,2}
				{3,1,2,4}
				{2,1,4,3}
				{2,4,1,3}
		\pregunta{Privatizar empresas, limitar el gasto social y romper con barreras proteccionistas, son características del modelo económico denominado} {null}
		\opciones{proteccionismo social}
				{desarrollo estabilizador}
				{sustitución de importaciones}
				{neoliberalismo económico}
				{milagro mexicano}
		\pregunta{Cultura que floreció en Monte Albán bajo el apelativo de "Gente de las nubes".}{null}
		\opciones{Teotihuacana}
				{Totonaca}
				{Zapoteca}
				{Olmeca}
				{Mexica}
				\pregunta{Producto determinante para la economía de la Nueva España}{null}
		\opciones{artesaniás}
				{metales}
				{henequén}
				{maíz}
				{café}
		\pregunta{En el siglo XIX, las reclamaciones monetarias de cuidadanos extranjeros afectados por las guerras civiles a conflictos internacionales como:}{null}		
		\opciones{la guerra con Estados Unidos de América}
				{la Primera Intervención Francesa}
				{la invasión de Isidro Barradas}
				{el conflicto de la Mesilla}
				{la Invasión Inglesa}
		\pregunta{Institución creada para recobar la legitimidad del régimen perdida en los comicios de 1998}{null}
		\opciones{Tribunal de lo Contensioso Electoral}
				{Partido de la revolución Democrática}
				{Instituto Federal Electoral}
				{Comités Cuidadanos}
				{Asamblea Constituyente}
		\pregunta{Proceso de modernización de México en donde se dan los primeros pasas hacia la centralización del Estado y el fortalecimiento del poder ejecutivo}{null}
		\opciones{Las leyes de reforma}
				{La venta de los bienes de la iglesia}
				{La formación del congreso constituyente}
				{El imperio de Maximiliano de Habsburgo}
				{Separación del estado y la iglesia}
				\pregunta{Ejemplo de los medios que permitieron a la iglesia alcanzar su influencia económica durante la colonia}{null}
		\opciones{Real patronato}
				{Donaciones}
				{Encomienda}
				{Prerrogativas}
		\pregunta{La minería fue una actividad de mucha importancia en la Nueva España, principalmente el los estados de:}{null}
		\opciones{Puebla,Oaxaca y Pachuca}
				{Guanajuato,Pachuca y Zacatecas}
				{San Luis Potosí, Guadalajara y Guanajuato}
				{Zacatecas,Michoacán y Guerrero}
		\pregunta{Después de la Guerra de Independencia, México hizo acuerdos de  libre comercio para recuperarse económicamente con:}{null}
		\opciones{Inglaterra y Francia}
				{Inglaterra y EU}
				{Alemania e Italia}
				{Cuba y Panamá}
				\pregunta{Son formas de gobierno y Administración de la Nueva España.
				I.Regencia
				II.Audiencia
				III.Principado
				IV.Virreinato
				V.Monarquía}{null}
		\opciones{I Y V}
				{II Y IV}
				{III Y V}
				{IV Y V}
				\pregunta{En lo económico el Porfiriato se caracterizó por:}{null}
		\opciones{abrir México a la inversión extranjera}
				{desarrollar la minería con recursos propios}
				{expropiar la industria del petróleo}
				{fomentar la pequeña propiedad agrícola}
				{fortalecer la participación del Estado en la economía}
		\pregunta{Las leyes de reforma promulgadas por Juárez establecieron:
		I.La nacionalización de los bienes eclesiásticos
		II.La religión católica como única
		III.El gobierno centralista
		IV.La superación de los fueros militar y eclesiástico
		V.La desaparición de la compañía de Jesus}{null}
		\opciones{I,II Y III}
				{I,III Y IV}
				{I,IV Y V}
				{III,IV Y V}
		\pregunta{Junto con la pobreza ¿Qué otros prblemas ha traido consigo el neoliberalismo y la globalización? 
		I.Deterioro ambietal
		II.Discriminación Social
		III.Terrorismo
		IV.Descolonización
		V.Intervencionismo estatal}{null}
		\opciones{I,II Y III}
				{I,II Y V}
				{II,III Y IV}
				{II,IV Y V}
				{III,IV Y V}
	 \pregunta{El plan de Ayala estableció entre sus postulados}{null}
		\opciones{desconnocer a Madero como presidente y devolver las tierras a los campesinos}
				{desconocer a Díaz como presidentey convocar al pueblo a rebelión}
				{reconocer al gobierno de León de la Barra y suspender los poderes legislativo y ejecutivo}
				{reconocer a Carranza como el primer jefe del ejército constitucionalista y convocar a elecciones}
		\end{multicols*}
	
\end{document}