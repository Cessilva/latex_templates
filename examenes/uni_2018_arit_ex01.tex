\documentclass[10pt,a4paper]{article}
	\usepackage{./../../miconfiguracion}
		\configPage

\begin{document}
	\BgThispage
	\makemytitle{Arítmetica}{Primer examen}
	\datosalu
	%\marcaagua
		
	%SECCION DE PREGUNTAS
	\begin{multicols*}{2}
	
	\pregunta{El valor de la siguiente operación es 
			$\mathbf{3\left[2^{-1}-\left(-\dfrac{3}{2}\right)\right]+2^0}$}
			{null}
		\opciones{$\dfrac{25}{2}$}
				{9}
				{7}
				{6}
				{$\dfrac{1}{2}$}

	\pregunta{Al simplificar $\sqrt[3]{54}$ se obtiene}
			{null}
		\opciones{$2\sqrt[3]{3}$}
				{$3\sqrt[3]{2}$}
				{$2\sqrt{3}$}
				{$3\sqrt{2}$}
	
	\pregunta{La expresión $\left(\dfrac{2^m}{2^{-3}}\right)^2$ 
			es equivalente a :}
			{null}
		\opciones{$2^{2m+6}$}
				{$2^{2m-6}$}
				{$2^{m^2-9}$}
				{$2^{2m+5}$}
	
	\pregunta{Al simplificar la expresión 
			$\left(\dfrac{3^4}{3^7}\right)^\frac{1}{3}$ se obtiene}
			{null}
		\opciones{$-3$}
			{$\dfrac{1}{3}$}
			{3}
			{$-\dfrac{1}{3}$}

	\pregunta{El número situado a la mitad entre
			$\dfrac{1}{6}$ y $\dfrac{1}{5}$ es:}
			{null}
		\opciones{$\dfrac{1}{4}$}
				{$\dfrac{11}{60}$}
				{$\dfrac{22}{29}$}
				{$2$}
				{$\dfrac{11}{15}$}

	\pregunta{ $-5-\dfrac{3}{4}\left[-8+5\left(\dfrac{2}{3}-2\right)\right]$ 
			se obtiene}
			{null}
		\opciones{0}
				{$\dfrac{35}{3}$}
				{$6$}
				{$-\dfrac{35}{3}$}
				{-50}

	\pregunta{En una tienda de abarrotes, un empleado vende 3/5 de 
			una pieza de jamón de pierna y después 6/8 del resto. 
			¿Cuánto de jamón quedan. si la pieza entera pesa 6.0 Kg?}
			{null}
		\opciones{3.3 kg}
				{2.4 kg}
				{0.60 kg}
				{2.7 kg}
				{0.9 kg}

	\pregunta{En la recta real, el número $\mathbf{\dfrac{7}{8}}$
			se encuentra entre los números}
			{null}
		\opciones{$\dfrac{11}{16}$ y $\dfrac{13}{16}$}
				{$\dfrac{15}{16}$ y $\dfrac{17}{16}$}
				{$\dfrac{25}{32}$ y $\dfrac{27}{32}$}
				{$\dfrac{27}{32}$ y $\dfrac{29}{32}$}
				{$\dfrac{53}{64}$ y $\dfrac{55}{64}$}

	\pregunta{En un contenedor de ferrocarril se cargan 120 metros cúbicos 
			de maíz. Si un metro cúbico de maíz pesa $\mathbf{\dfrac{5}{6}}$ 
			de tonelada, ¿cuál es el peso de la carga en el contenedor?}
			{null}
		\opciones{100 toneladas}
				{110 toneladas}
				{120 toneladas}
				{120.83 toneladas}
				{144 toneladas}

	\pregunta{Al efectuar la multiplicación de 
			$\mathbf{(\sqrt{2})(\sqrt{2})}$ se obtiene}
			{null}

		\opciones{$\sqrt{2}$}
			{2}
			{$2\sqrt{2}$}
			{4}
			{1}

	\pregunta{Determinar el m.c.m de 60,42 y 12 en términos de números primos}
			{null}
		\opciones{$2^2 \times 3 \times 5 \times 7$}
				{$2 \times 3 \times 5 \times 7$}
				{$2 \times 3^2 \times 5 \times 7$}
				{$2 \times 3 \times 5^2 \times 7$}
				{$2 \times 3 \times 5 \times 7^2$}
	
	\pregunta{Si una motocicleta gasta un litro de gasolina por cada 18 
			kilómetros de recorrido y otra gasta un litro por cada 16 kilómetros, 
			¿cuál es el menor número entero de litros de gasolina que deben tener
			en sus tanques las motocicletas para que recorran exactamente la misma
			distancia?}
			{null}
		\opciones{5 y 6}
				{7 y 8}
				{8 y 9}
				{9 y 12}
				{9 y 10}

	\pregunta{ Calcula $7-3\{5-(2-6)+4[9-(10+1)-(4+7)]+6-8\}+10$}
			{null}
		\opciones{-142}
				{-160}
				{-674}
				{1481}
				{152}

	\pregunta{El valor de la expresión $|(-4)^2\cdot(-6)^3+(-1)^4$ es:}
			{null}
		\opciones{-3456}
				{3455}
				{-3455}
				{3452}

	\pregunta{Resuelve la siguiente operación: 
			$\dfrac{2}{2+\dfrac{1}{2+\dfrac{1}{2}}}$}
			{null}
		\opciones{$\frac{5}{6}$}
				{$\frac{3}{2}$}
				{$\frac{1}{2}$}
				{$\frac{2}{5}$}
				{$\frac{3}{4}$}
	\pregunta{Si un vestido cuesta \$ 347.30 con IVA incluido
			entonces el precio del vestido sin el 15 \% del IVA es:}
			{null}
		\opciones{\$296}
				{\$300}
				{\$295.20}
				{\$302}
				{\$305.20}

	\pregunta{El valor de la operación es 
			$-5^2-5^2-5^2-5^2-5^2$ es igual a:}
			{null}
		\opciones{100}		
				{-125}
				{625}
				{125}
				{-625}

	\pregunta{Identifica un número irracional de los siguientes}
			{null}
		\opciones{5.111111111...}
				{4.011011011011}
				{$\sqrt{5}$}
				{3.04}
				{1, 000, 000}

	\pregunta{La razón entre los números de programas respecto a las repeticiones
			en T.V. por cable es 2 a 27. Si Carlos contó solamente 8 nuevos programas
			una noche ¿Cuántas repeticiones hubo?}
			{null}
		\opciones{7}
				{4}
				{108}
				{110}
				{62}
	\pregunta{El resultado simplificado de $\dfrac{2}{3}+\dfrac{7}{8}\div\dfrac{7}{2}-\dfrac{1}{6} $}
			{null}
		\opciones{$\dfrac{57}{16}$}
				{2}
				{$-\dfrac{7}{16}$}
				{$\dfrac{4}{3}$}
				{$\dfrac{3}{4}$}
	
	
	\end{multicols*}
	
\end{document}