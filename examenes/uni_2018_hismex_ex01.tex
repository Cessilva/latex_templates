 \documentclass[10pt,a4paper]{article}
	\usepackage{../miconfiguracion}
  		\configPage

\begin{document}
	\BgThispage
	\makemytitle{Historia de México.}{Primer examen}
	\datosalu
		
	%SECCION DE PREGUNTAS
	\begin{multicols*}{2}
	
	\pregunta{Los aztecas crearon las guerras floridas con el propósito de obtener prisioneros destinados a:}
			{null}
		\opciones{los sacrificios rituales}
				{los trabajos agricolas}
				{la vida militar }
				{el combate de los españoles}
				{la custodia de los templos}	
				\pregunta{¿Cuál de los hechos tuvo lugar antes de la caida de Tenochtitlan?}
			{null}
		\opciones{la publicación de la obra de Bernal Díaz del Castillo}
				{los viajes de Alehandro Von Humboldt}
				{los viajes de Cristóbal Colón}
				{el primer viaje de la Nao de China}
				{el reparto de las encominedas}
				\pregunta{¿Cuál era el grupo social privilegiado en la Nueva España?}
			{null}
		\opciones{criollos}
				{mestizos}
				{indios}
				{peninsulares}
				{mulatos}
				\pregunta{Ordena cronológicamente los acontecimientos relacionados con la independencia.
				1: Consumación de la independencia
				2: Constitución de Apatzingán
				3: Llegada de Mina a México
				4: Grito de dolores}
			{null}
		\opciones{2,4,3,1}
				{3,2,1,4}
				{4,1,2,3}
				{1,4,2,3}
				{4,2,3,1}
				\pregunta{Uno de los acontecimientos que precipitó el inicio del movimiento de independencia en la nueva españa fue la:}
			{null}
		\opciones{muerte de Fernando VII}
				{derrota de armada invensible española}
				{invasión napoleónica a España}
				{Guerra de sucesión Española}
				{separación de Portugal del reino español}
				\pregunta{¿Cuál fue una de las influencias externas que favoreció el movimiento de independencia de la Nueva España?}
			{null}
		\opciones{la Revolución Francesa}
				{el Socialismo Utópico}
				{la Doctrina Monroe}
				{la reforma Luterana}
				{el Renacimiento}
				\pregunta{Ante la invasión napoleónica a España, el Ayuntamiento de la Cuidad de México decidió:}
			{null}
		\opciones{apoyar el reinado de José Bonaparte}
				{no intervenir en los sucesos de España}
				{declarar fidelidad a Carlos IV}
				{jurar a Fernando VII como rey legítimo}
				{enviar tropas para combatir a los franceses}
				\pregunta{Indica el nombre de las organizaciones semi-secretas que intervinieron en la política de México recién independizado.}
			{null}
		\opciones{los clubes liberales}
				{el clero secular}
				{las logias masónicas}
				{los grupos borbonista}
				{las confresiones protestanes}
				\pregunta{Personaje que ocupó varias veces la presidencia entre 1833 y 1855.} {null}
		\opciones{Antonio López de Sanya Anna}
				{Valentín Gomez Farías}
 				{Porfirio Diaz}
				{Anastasio Bustamante}
				{Benito Juárez}
				\pregunta{Durante el Porfiriato, el sector de la infraestructura donde invitieron principalmente los capitalistas extranjeros fue el de:}
			{null}
		\opciones{carreteras}
				{ferrocarriles}
				{puertos}
				{teléfrafos}
				{aeropuertos}
				\pregunta{Una de las acciones de Lázaro Cárdenas para llevar a cabo la Reforma Agraria fue la creación}
			{null}
		\opciones{de la Confederación Nacional Campesina}
				{de las Comisiones Nacionales de Irrigación y de Caminos}
				{del primer Código Agrario.}
				{del Banco de Crédito agrícola}
				{la Unión de Campesinos Unidos}
				\pregunta{Proceso de modernización de México en donde los primeros pasos hacia la centralización del Estado y el fortalecimiento del poder ejecutivo.}
			{null}
		\opciones{Las leyes de reforma}
				{La venta de los bienes de la iglesia}
				{La reformación del congreso constituyente}
				{El imperio de Maximiliano de Habsburgo}
				{Tratados de la soledad}
				\pregunta{Autor representativo del Modernismo literario en México}
			{null}
		\opciones{Joaquín Fernández de Lizardi}
				{Manuel Payno}
				{Manuel Gutiérrez Nájera}
				{Guillermo Prieto}
				{Guillermo Prieto}
				{Carlos Fuentes}
				\pregunta{Institución que se crea para recobrar la lefitimidad del régimen perdida en los comicios de 1988}
			{null}
		\opciones{Tribunal de lo Contencioso Electoral}
				{Partido de la Revolución Democrática}
				{Instituto Federal Electoral}
				{Comités de los cuidadanos}
				{Partido socialista mexicano}
				\pregunta{Una característica destacada de la economía durante el porfiriato fue que}
			{null}
		\opciones{estuvo dominada por extrajeros}
				{beneficiaba a la mayor parte de la población}
				{estaba dedicada exclusivamente a la exportación}
				{se refujeron las importaciones}
				{se nacionalizó la banca }
				\pregunta{El plan de Ayala establecío entre sus postulados}
			{null}
		\opciones{desconocer a Madero como presidente y devolver a los pueblos sus tierras}
				{desconocer a Díaz como presidente y convocar al pueblo a la revelión }
				{desconocer a gobierno de Huerta y establecer in régimen Constitucional}
				{reconocer a Carranza como el primer jefe del ejército constitucionaista y convocar a elecciones}
				{desconocer a Madero como presidente y llamar al pueblo a la guerra.}
				\pregunta{Las consecuencias económicas de la de la instauración del modelo neoliberal en México son}
			{null}
		\opciones{democrarización y libre comercio}
				{descentralización y reforma del estado}
				{privatización y libre comercio}
				{libertad de expresión y participación cuidadana}
				{reforma constitucional y globalización}
				\pregunta{Para fortalecer la transparencia y la legalidad en los procesos democráticosde México, desde 1990 el congreso de la Nación.... }
			{null}
		\opciones{permitió el registro del EZLN como partido político}
				{ordenó que la elección presidencial se realizara cada seis años}
				{discutió y aprobó el Código Federal Institucional y Procedimientos Electorales}
				{derogó al COFIPE y creó al Tribunal Federal Electoral como máxima  instancia}
				{autorizó al Instituto Federal Electoral para nombrar el presidente de la nación}
				\pregunta{Autorizó la construcción de Cuidad Universitaria}
			{null}
		\opciones{Adolfo López Mateos}
				{Miguel Alemán Valdés}
				{Manuel Ávila Camacho}
				{Adolfo Ruiz Cortines}
				{Carlos Salinas de Gortari}
				\pregunta{Presidente mexicano que firmó el TLCAN}
			{null}
		\opciones{Carlos Sainas de Gortari}
				{Manuel Ávila Camacho}
				{Luis Echeverría Álvares}
				{Ernesto Cedillo Ponce de León}
				{Miguel de la Madrid Hurtado}	
	\end{multicols*}
	
\end{document}