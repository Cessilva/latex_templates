\documentclass{mylib/reporte}
\usepackage{float}
\graphicspath{ {img/dise/} }
\usepackage{multicol}
\title{Reporte}
\author{rodrigofranciscopablo }

\subject{Diseño Digital Moderno}
\mytitle{Tarea 3}
\mysubTitle{Conceptos}
\students{
	Francisco Pablo \textsc{Rodrigo}
}
\teacher{Ing. Mandujano Wild \textsc{Roberto F.}}
\group{6}
\deliverDate{febrero 2019}

\begin{document}

\coverPage

\tableofcontents

\section{¿Qué es quilo y quimo?}

\subsection{Quilo}

Líquido blanco lechoso que se forma en el intestino delgado durante la digestión. Está compuesto de líquido linfático y grasas. Hay vasos linfáticos especiales que transportan el quilo desde los intestinos a la sangre.

\subsection{Quimo}

Masa homogénea en que se transforman los alimentos dentro del estómago por efecto de la digestión.

\section{¿Qué es el SENACYT y el CENACE?}

\subsection{SENACYT}

La Secretaría Nacional de Ciencia, Tecnología e Innovación tiene como objetivo fortalecer, apoyar, inducir y promover el desarrollo de la ciencia, la tecnología y la innovación con el propósito de elevar el nivel de productividad, competitividad y modernización en el sector privado, el gobierno, el sector académico investigativo, y la población en general.

\subsection{CENACE}

El Centro Nacional de Control de Energía (CENACE) es un organismo público descentralizado cuyo objeto es ejercer el Control Operativo del Sistema Eléctrico Nacional; la Operación del Mercado Eléctrico Mayorista y garantizar imparcialidad en el acceso a la Red Nacional de Transmisión y a las Redes Generales de Distribución.

Como Operador Independiente del Sistema realiza sus funciones bajo los principios de eficiencia, transparencia y objetividad, cumpliendo los criterios de calidad, confiabilidad, continuidad, seguridad y sustentabilidad en la operación y control del Sistema Eléctrico Nacional.

Realiza la operación del Mercado Eléctrico Mayorista en condiciones que promueven la competencia, eficiencia e imparcialidad, mediante la asignación y despacho óptimos de las Centrales Eléctricas para satisfacer la demanda de energía del Sistema Eléctrico Nacional.

Es responsable de formular los programas de ampliación y modernización de la Red Nacional de Transmisión y de las Redes Generales de Distribución, los cuales en caso de ser autorizados por la Secretaria de Energía (SENER) se incorporan al Programa de Desarrollo del Sistema Eléctrico Nacional (PRODESEN).


\section{¿Cómo se escribe el segundo método de minimización?}

 Quine McCluskey


\end{document}
