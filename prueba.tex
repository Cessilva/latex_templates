\documentclass[10pt,a4paper]{article}
	\usepackage{miconfiguracion}
		\configPage

\begin{document}
	\BgThispage
	\makemytitle{Segundo examen}{UNAM}
	\datosalu
	%\marcaagua
		
	%SECCION DE PREGUNTAS
	\begin{multicols*}{2}
	
	\section{Bilogía}
	
	\pregunta{Investigador que le asigno el nombre a la celula}{null}
	\opciones{Rudolf Virchow}
		{Robert Brown}
		{Pasteur}
		{Robert Hooke}	
	% A. Robert Hook 
	
	\pregunta{Organelo cuya funcion es transportar moleculas intracelularmente}
	{null}
	\opciones{Membrana y ncuelo}
		{Mitocondrias y nucleo}
		{Ribosomas y cloroplastos}
		{Aparato de Golgi y reticulo endoplasmatico}
		
	\pregunta{En la fosotintesis la reaccion no dependiente de luz se lleva acabo en:}{null}
	\opciones{Mitocondrias}
		{Tilacoides}
		{Grana} 
		{Estroma}
		
	\pregunta{Producto final del glucolisis}{null}
	\opciones{2 ATP, 2 NADH, 2 Piruvatos}
		{3 ATP, 3 NADH, 3 Piruvatos}
		{6 ATP, 2 FADH, 2 Piruvatos}
		{2 ATP, 2 FADH, 1 Piruvato}

	\pregunta{en la actualidad es la teoría mas aceptada sobre el origen de los sere vivos}{null}
	\opciones{Creacionista}
	{Endosimbiótica}
	{Vitalismo}
	{Quimiosintética}

	\pregunta{Cual es el orden de fases de la mitosis}{null}
	\opciones{Profase, Metafase, Telofase, Anafase}
		{Anafase, Profase, Metafase, Telofase}
		{Profase, Telofase, Metafase, Anafase}
		{Metafase, Anafase, Telofase, Profase}
		
	\pregunta{Numero de cromosomas que posen los gametos y las células somáticas}{null}
	\opciones{22 y 38}
		{23 y 36}
		{24 y 37}
		{36 y 23}
		
	\pregunta{El sexo cromosómico se establece durante el proceso de}{null}
	\opciones{Ovulación}
		{Segmentación}
		{Implantación}
		{Fecundación}
		
	\pregunta{Una célula somática presenta un número cromosómico con 46 XX; durante la mitosis, al dividirse, es de esperarse que forme células con }{null}
	\opciones{23 X}
		{23 XX.}
		{46 XX.}
		{46 XY.}
		
	\pregunta{Relaciona las fases del ciclo celular con los procesos que ocurren en cada una de ellas.\\
Fases\\
	I. Mitosis.\\
	II. Interfase.\\
	Procesos\\
	a. Duplicación del ADN.\\
	b. Crecimiento de la célula.\\
	c. Síntesis de proteínas.\\
	d. División celular.}{null}
	\opciones{I: a – II: b, c, d}
		{I: d – II: a, b, c}
		{I: c, d – II: a, b}
		{I: b, c – II: a, d}
		
	\section{QUIMICA}

	\pregunta{¿Cuál es la molaridad de una disolución que contiene 20g de NaOH en 2L de solución?\\
	Masa atómica:  Na: 23	O: 16	H:1}{null}
	\opciones{0.25M}
		{0.50M}
		{0.75M}
		{1.00M}

	\pregunta{¿En cuál de las siguientes opciones hay materiales formados únicamente por elementos?}{null}
	\opciones{Na(g), Cl2(g), P4(s)}
		{O2(g), He(g), CO(g)}
		{S8(g), N2(g), SO2(g)}
		{CO(g), Na(s), S8(s)}

	\pregunta{¿Cuál reacción representa la formación de una sal?}{null}
	\opciones{SO2 + H2O → H2SO3}
		{Cl2 + H2O → HCl}
		{N2O5 + H2O → 2HNO3}
		{HCl + KOH →KCl + H2O}

	\pregunta{¿Cuál de los siguientes valores de pH corresponde a la mayor concentración de iones OH–?}{null}
	\opciones{2}
		{7}
		{8}
		{13}
	\pregunta{Relaciona los compuestos con la función que les corresponde.\\
Compuestos\\
	 I. LiOH\\
	II. H3PO4\\
	III.NaH \\
	Funciones\\
	a. Ácido.\\
	b. Hidróxido.\\
	c. Sal. }{null}
	\opciones{I:c – II:b –III:a}
		{I:b – II:c –III:a}
		{I:b – II:a – III:c}
		{I:a – II:c –III:b}

	\pregunta{¿Cuál es el enunciado verdadero? }{null}
	\opciones{El aire es un compuesto y el cloruro de sodio es una mezcla.}
		{El cloruro de sodio es un elemento y la plata es un compuesto.} 
		{El aire es una mezcla y la plata es un compuesto. }
		{El cloruro de sodio es un compuesto y el aire es una mezcla.}

	\pregunta{Al enlace que une a las moléculas de agua se le denomina }{null}
	\opciones{covalente. }	
		{iónico. }
		{coordinado. }
		{puente de hidrógeno.}

	\pregunta{En el aire que respiramos, el elemento gaseoso que se encuentra en mayor cantidad es el }{null}	

	\opciones{hidrógeno. }
		{nitrógeno. }
		{oxígeno. }
		{ozono}

	\pregunta{La función principal de un catalizador es favorecer que }{null}
	\opciones{aumente la cantidad de reactivos sin reaccionar.}
		{los productos tengan mayor pureza. }
		{los reactivos se consuman más rápido.}
		{aumente la temperatura de los reactivos.}

	\pregunta{Una proteína está formada por }{null}
	\opciones{una serie de enzimas. }
		{una cadena de aminoácidos. }
		{un polímero de carbohidratos. }
		{un conjunto de triglicéridos.}
		
	\section{Literatura}
	
	\pregunta{Elige la función de la lengua que predomina en el siguiente ejemplo. Luisa, ¿puedes limpiar la mesa y lavar los trastes por favor? }{null}
	\opciones{Metalingüística.}	
		{Apelativa.}
		{Referencial.}
		{Sintomática.}
		
	\pregunta{Identifica el enunciado en el que la lengua está usada en su función poética. }{null}
		\opciones{Era apenas una niña cuando la vi por primera vez.}
			{A las tres en punto moriría un transeúnte.}
			{Las nieves del tiempo platearon mi sien. }
			{Chopin soñó que estaba muerto en el lago}
			
	\pregunta{¿Qué modo discursivo predomina en el siguiente párrafo?
 El alcoholismo es una enfermedad progresiva y crónica, que presenta síntomas que van desde el malestar hasta el dolor intenso. Depende de varios factores, principalmente de la predisposición genética y de la influencia del medio ambiente familiar y social. Pese a que afecta todo el cuerpo y provoca una variedad de problemas médicos, los principales síntomas se manifiestan en el sistema nervioso. A través de éste, en especial del cerebro, la adicción produce diversos trastornos en el pensamiento, las emociones y la conducta del enfermo. }{null}
 		\opciones{Instrucción.}
 			{Descripción. }
 			{Enumeración. }
 			{Explicación.}
 			
 	\pregunta{¿Qué modo discursivo predomina en el siguiente ejemplo? 
El libro comprende tres capítulos, con cinco subtemas cada uno. Sin embargo, no tiene consistencia. Esto se corrobora, en primer lugar, porque carece de un apartado de conclusiones. En segundo lugar, no cita las fuentes bibliográficas en las que se apoya. Esto hace que el texto sea de poco fiar. }{null}
		\opciones{Enumeración. }
			{Descripción. }
			{Argumentación.}
			{Narración.}
			
	\pregunta{¿En qué versos del siguiente poema de Sor Juana Inés de la Cruz aparece una metáfora? \\
Al que ingrato me deja, busco amante; 1 \\
 al que amante me sigue, dejo ingrata; 2 \\
constante adoro a quien mi amor maltrata; 3 \\
 maltrato a quien mi amor busca constante. 4 \\
Al que trato de amor, hallo diamante, 5 \\
y soy diamante al que de amor me trata; 6 \\
triunfante quiero ver al que me mata, 7 \\
y mato al que me quiere ver triunfante. 8 \\
Si a éste pago, padece mi deseo; 9 \\
si ruego a aquél, mi pundonor enojo: 10 \\
de entre ambos modos infeliz me veo. 11 \\
Pero yo, por mejor partido escojo 12 \\
de quien no quiero, ser violento empleo, 13 \\
que, de quien no me quiere, vil despojo. 14  \\}{null}
		\opciones{1, 3 y 7 }
			{5 y 6 }
			{7 y 14 }
			{1 y 2}
	\pregunta{Elige las características del poema lírico. }{null}
	\opciones{Objetividad, profundidad y extensión. }
		{Argumentación, objetividad y ejemplificación. }
		{Individualismo y subjetividad. }
		{Veracidad, exactitud y desenlace}
	
	\pregunta{Poeta mexicano de la segunda mitad del siglo XX, ganador del premio Nobel. }{null}
	\opciones{Carlos Fuentes. }
		{Jaime Sabines. }
		{Octavio Paz. }
		{Carlos Monsiváis.}
	
	\pregunta{Un cuento se diferencia de una novela porque éste tiene }{null}
	\opciones{amplio desarrollo psicológico de los personajes.}
		{desarrollo elaborado y rápido desenlace. }
		{brevedad y rápido desenlace. }
		{intensidad y múltiples hilos narrativos}
			
	\pregunta{Movimiento literario que surge en la segunda mitad del siglo XIX, como reacción ante el individualismo extremo y el idealismo que caracterizó al Romanticismo.}{null}
	\opciones{Neoclasicismo. }
		{Vanguardismo. }
		{Realismo. }
		{Surrealismo}
	
	\end{multicols*}
	
\end{document}